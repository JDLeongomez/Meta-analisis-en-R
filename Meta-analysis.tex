% Options for packages loaded elsewhere
\PassOptionsToPackage{unicode}{hyperref}
\PassOptionsToPackage{hyphens}{url}
\PassOptionsToPackage{dvipsnames,svgnames,x11names}{xcolor}
%
\documentclass[
  bookmarksnumbered]{article}
\usepackage{amsmath,amssymb}
\usepackage{lmodern}
\usepackage{iftex}
\ifPDFTeX
  \usepackage[T1]{fontenc}
  \usepackage[utf8]{inputenc}
  \usepackage{textcomp} % provide euro and other symbols
\else % if luatex or xetex
  \usepackage{unicode-math}
  \defaultfontfeatures{Scale=MatchLowercase}
  \defaultfontfeatures[\rmfamily]{Ligatures=TeX,Scale=1}
\fi
% Use upquote if available, for straight quotes in verbatim environments
\IfFileExists{upquote.sty}{\usepackage{upquote}}{}
\IfFileExists{microtype.sty}{% use microtype if available
  \usepackage[]{microtype}
  \UseMicrotypeSet[protrusion]{basicmath} % disable protrusion for tt fonts
}{}
\makeatletter
\@ifundefined{KOMAClassName}{% if non-KOMA class
  \IfFileExists{parskip.sty}{%
    \usepackage{parskip}
  }{% else
    \setlength{\parindent}{0pt}
    \setlength{\parskip}{6pt plus 2pt minus 1pt}}
}{% if KOMA class
  \KOMAoptions{parskip=half}}
\makeatother
\usepackage{xcolor}
\IfFileExists{xurl.sty}{\usepackage{xurl}}{} % add URL line breaks if available
\IfFileExists{bookmark.sty}{\usepackage{bookmark}}{\usepackage{hyperref}}
\hypersetup{
  pdftitle={Meta-análisis de correlaciones en R:},
  pdfauthor={Juan David Leongómez },
  colorlinks=true,
  linkcolor={iacoldark},
  filecolor={Maroon},
  citecolor={Blue},
  urlcolor={blue},
  pdfcreator={LaTeX via pandoc}}
\urlstyle{same} % disable monospaced font for URLs
\usepackage[margin=2cm]{geometry}
\usepackage{color}
\usepackage{fancyvrb}
\newcommand{\VerbBar}{|}
\newcommand{\VERB}{\Verb[commandchars=\\\{\}]}
\DefineVerbatimEnvironment{Highlighting}{Verbatim}{commandchars=\\\{\}}
% Add ',fontsize=\small' for more characters per line
\usepackage{framed}
\definecolor{shadecolor}{RGB}{237,237,237}
\newenvironment{Shaded}{\begin{snugshade}}{\end{snugshade}}
\newcommand{\AlertTok}[1]{\textcolor[rgb]{0.75,0.01,0.01}{\textbf{\colorbox[rgb]{0.97,0.90,0.90}{#1}}}}
\newcommand{\AnnotationTok}[1]{\textcolor[rgb]{0.79,0.38,0.79}{#1}}
\newcommand{\AttributeTok}[1]{\textcolor[rgb]{0.00,0.34,0.68}{#1}}
\newcommand{\BaseNTok}[1]{\textcolor[rgb]{0.69,0.50,0.00}{#1}}
\newcommand{\BuiltInTok}[1]{\textcolor[rgb]{0.39,0.29,0.61}{\textbf{#1}}}
\newcommand{\CharTok}[1]{\textcolor[rgb]{0.57,0.30,0.62}{#1}}
\newcommand{\CommentTok}[1]{\textcolor[rgb]{0.54,0.53,0.53}{#1}}
\newcommand{\CommentVarTok}[1]{\textcolor[rgb]{0.00,0.58,1.00}{#1}}
\newcommand{\ConstantTok}[1]{\textcolor[rgb]{0.67,0.33,0.00}{#1}}
\newcommand{\ControlFlowTok}[1]{\textcolor[rgb]{0.12,0.11,0.11}{\textbf{#1}}}
\newcommand{\DataTypeTok}[1]{\textcolor[rgb]{0.00,0.34,0.68}{#1}}
\newcommand{\DecValTok}[1]{\textcolor[rgb]{0.69,0.50,0.00}{#1}}
\newcommand{\DocumentationTok}[1]{\textcolor[rgb]{0.38,0.47,0.50}{#1}}
\newcommand{\ErrorTok}[1]{\textcolor[rgb]{0.75,0.01,0.01}{\underline{#1}}}
\newcommand{\ExtensionTok}[1]{\textcolor[rgb]{0.00,0.58,1.00}{\textbf{#1}}}
\newcommand{\FloatTok}[1]{\textcolor[rgb]{0.69,0.50,0.00}{#1}}
\newcommand{\FunctionTok}[1]{\textcolor[rgb]{0.39,0.29,0.61}{#1}}
\newcommand{\ImportTok}[1]{\textcolor[rgb]{1.00,0.33,0.00}{#1}}
\newcommand{\InformationTok}[1]{\textcolor[rgb]{0.69,0.50,0.00}{#1}}
\newcommand{\KeywordTok}[1]{\textcolor[rgb]{0.12,0.11,0.11}{\textbf{#1}}}
\newcommand{\NormalTok}[1]{\textcolor[rgb]{0.12,0.11,0.11}{#1}}
\newcommand{\OperatorTok}[1]{\textcolor[rgb]{0.12,0.11,0.11}{#1}}
\newcommand{\OtherTok}[1]{\textcolor[rgb]{0.00,0.43,0.16}{#1}}
\newcommand{\PreprocessorTok}[1]{\textcolor[rgb]{0.00,0.43,0.16}{#1}}
\newcommand{\RegionMarkerTok}[1]{\textcolor[rgb]{0.00,0.34,0.68}{\colorbox[rgb]{0.88,0.91,0.97}{#1}}}
\newcommand{\SpecialCharTok}[1]{\textcolor[rgb]{0.24,0.68,0.91}{#1}}
\newcommand{\SpecialStringTok}[1]{\textcolor[rgb]{1.00,0.33,0.00}{#1}}
\newcommand{\StringTok}[1]{\textcolor[rgb]{0.75,0.01,0.01}{#1}}
\newcommand{\VariableTok}[1]{\textcolor[rgb]{0.00,0.34,0.68}{#1}}
\newcommand{\VerbatimStringTok}[1]{\textcolor[rgb]{0.75,0.01,0.01}{#1}}
\newcommand{\WarningTok}[1]{\textcolor[rgb]{0.75,0.01,0.01}{#1}}
\usepackage{longtable,booktabs,array}
\usepackage{calc} % for calculating minipage widths
% Correct order of tables after \paragraph or \subparagraph
\usepackage{etoolbox}
\makeatletter
\patchcmd\longtable{\par}{\if@noskipsec\mbox{}\fi\par}{}{}
\makeatother
% Allow footnotes in longtable head/foot
\IfFileExists{footnotehyper.sty}{\usepackage{footnotehyper}}{\usepackage{footnote}}
\makesavenoteenv{longtable}
\usepackage{graphicx}
\makeatletter
\def\maxwidth{\ifdim\Gin@nat@width>\linewidth\linewidth\else\Gin@nat@width\fi}
\def\maxheight{\ifdim\Gin@nat@height>\textheight\textheight\else\Gin@nat@height\fi}
\makeatother
% Scale images if necessary, so that they will not overflow the page
% margins by default, and it is still possible to overwrite the defaults
% using explicit options in \includegraphics[width, height, ...]{}
\setkeys{Gin}{width=\maxwidth,height=\maxheight,keepaspectratio}
% Set default figure placement to htbp
\makeatletter
\def\fps@figure{htbp}
\makeatother
\setlength{\emergencystretch}{3em} % prevent overfull lines
\providecommand{\tightlist}{%
  \setlength{\itemsep}{0pt}\setlength{\parskip}{0pt}}
\setcounter{secnumdepth}{5}
\newlength{\cslhangindent}
\setlength{\cslhangindent}{1.5em}
\newlength{\csllabelwidth}
\setlength{\csllabelwidth}{3em}
\newlength{\cslentryspacingunit} % times entry-spacing
\setlength{\cslentryspacingunit}{\parskip}
\newenvironment{CSLReferences}[2] % #1 hanging-ident, #2 entry spacing
 {% don't indent paragraphs
  \setlength{\parindent}{0pt}
  % turn on hanging indent if param 1 is 1
  \ifodd #1
  \let\oldpar\par
  \def\par{\hangindent=\cslhangindent\oldpar}
  \fi
  % set entry spacing
  \setlength{\parskip}{#2\cslentryspacingunit}
 }%
 {}
\usepackage{calc}
\newcommand{\CSLBlock}[1]{#1\hfill\break}
\newcommand{\CSLLeftMargin}[1]{\parbox[t]{\csllabelwidth}{#1}}
\newcommand{\CSLRightInline}[1]{\parbox[t]{\linewidth - \csllabelwidth}{#1}\break}
\newcommand{\CSLIndent}[1]{\hspace{\cslhangindent}#1}
\usepackage{float} \floatplacement{figure}{H} \usepackage[utf8]{inputenc} \usepackage{fancyhdr} \pagestyle{fancy} \lhead{Juan David Leongómez} \rhead{\textit{Meta-análisis de correlaciones en {R:} Guía práctica}} \rfoot{\footnotesize{{doi:} \href{https://doi.org/10.5281/zenodo.5640182}{10.5281/zenodo.5640182}}} \lfoot{\footnotesize{Versión 2}} \renewcommand{\abstractname}{Descripción} \usepackage[spanish]{babel} \usepackage{hanging} \usepackage{amsthm,amssymb,amsfonts} \usepackage{tikz,lipsum,lmodern} \usepackage[most]{tcolorbox} \usepackage{fontawesome5} \usepackage{svg} \renewcommand\spanishtablename{Tabla} \usepackage{multirow,booktabs,setspace,caption} \DeclareCaptionLabelSeparator*{spaced}{\\[1ex]} \DeclareCaptionLabelSeparator{point}{. } \captionsetup[table]{labelfont=bf,textfont=it,format=plain,justification=justified, singlelinecheck=false,labelsep=spaced,skip=5pt} \captionsetup[figure]{labelfont=bf,format=plain,justification=justified, singlelinecheck=false,labelsep=point,skip=5pt} \usepackage{orcidlink} \definecolor{iacol}{RGB}{246, 130, 18} \definecolor{iacoldark}{RGB}{246, 100, 18}
\usepackage{booktabs}
\usepackage{longtable}
\usepackage{array}
\usepackage{multirow}
\usepackage{wrapfig}
\usepackage{float}
\usepackage{colortbl}
\usepackage{pdflscape}
\usepackage{tabu}
\usepackage{threeparttable}
\usepackage{threeparttablex}
\usepackage[normalem]{ulem}
\usepackage{makecell}
\usepackage{xcolor}
\ifLuaTeX
  \usepackage{selnolig}  % disable illegal ligatures
\fi

\title{Meta-análisis de correlaciones en R:}
\usepackage{etoolbox}
\makeatletter
\providecommand{\subtitle}[1]{% add subtitle to \maketitle
  \apptocmd{\@title}{\par {\large #1 \par}}{}{}
}
\makeatother
\subtitle{Guía práctica}
\author{Juan David Leongómez \orcidlink{0000-0002-0092-6298}\textsuperscript{}}
\date{27 febrero, 2022}

\begin{document}
\maketitle

\newtcolorbox[auto counter]{ROut}[2][]{
                lower separated=false,
                colback=white,
                colframe=iacol,
                fonttitle=\bfseries,
                colbacktitle=iacol,
                coltitle=black,
                boxrule=1pt,
                sharp corners,
                breakable,
                enhanced,
                attach boxed title to top left={yshift=-0.1in,xshift=0.15in},
                boxed title style={boxrule=0pt,colframe=white,},
              title=#2,#1}

\begin{center}
Laboratorio de Análisis del Comportamiento Humano (LACH), Facultad de Psicología, Universidad El Bosque, Bogotá, Colombia. Email: \href{mailto:jleongomez@unbosque.edu.co}{jleongomez@unbosque.edu.co}. Web: \href{https://jdleongomez.info/es}{jdleongomez.info}.

\vfill
\textbf{Descripción}
\end{center}

\par
\begingroup
\leftskip3em
\rightskip\leftskip

Esta guía contiene todo el código explicaciones básicas, paso a paso, para hacer un meta-análisis de coeficientes de correlación en R, usando los paquetes \href{https://www.metafor-project.org/doku.php}{\texttt{metafor}} (\protect\hyperlink{ref-viechtbauer2010}{Viechtbauer, 2010}) y \href{https://cran.r-project.org/web/packages/metaviz/vignettes/metaviz.html}{\texttt{metaviz}} (\protect\hyperlink{ref-KossmeierMetaviz}{Kossmeier et al., 2020}). Está principalmente basado en \href{https://youtu.be/lH4VZMTEZSc}{este video}, creado por Daniel S. Quintana (\protect\hyperlink{ref-quintanaHowPerformMetaanalysis2021}{2021}), pero contiene citas a fuentes primarias, además de información que he agregado.

Esta guía asume un manejo básico de R, así como una comprensión básica del meta-análisis. Sin embargo, de ser necesario y como introducción, recomiendo ver \href{https://youtu.be/ntBbkOn9D_o}{este video introductorio} al meta-análisis en \emph{jamovi} (\protect\hyperlink{ref-leongomezMetaanalysis2021}{Leongómez, 2021}) que publiqué anteriormente en mi canal de YouTube \href{https://www.youtube.com/c/InvestigaciónAbierta}{\emph{Investigación Abierta}}.

\href{https://www.youtube.com/c/InvestigaciónAbierta}{\includegraphics{images/Logo-IA-Rectangulo.pdf}}

\par
\endgroup
\vfill

\textbf{Cita esta guía como } \hrulefill 

\begin{hangparas}{.25in}{1}
Leongómez, J. D. (2022). \textit{Meta-análisis de correlaciones en R: Guía práctica.} Zenodo. \url{https://doi.org/10.5281/zenodo.5640182}
\end{hangparas}

\newpage
{\hypersetup{hidelinks}
 \setcounter{tocdepth}{5}
 \tableofcontents
}
\newpage

\hypertarget{base-de-datos-de-ejemplo}{%
\section{Base de datos de ejemplo}\label{base-de-datos-de-ejemplo}}

Para los ejemplos usados en ésta guía, usaré la base de datos \texttt{dat.molloy2014}, tomada de Molloy et al. (\protect\hyperlink{ref-molloy2013}{2013}).

Esta base de datos viene incluida con el paquete \texttt{\{metafor\}} de R. Básicamente, Molloy et al. (\protect\hyperlink{ref-molloy2013}{2013}) estudiaron si existe una asociación entre la diligencia (\emph{conscientiousness}) y la adherencia a la medicación. En otras palabras, ¿las personas más diligentes son más propensas a cumplir con la medicación prescrita?

Primero, primero, debemos cargar los paquetes que usaremos, incluyendo \texttt{\{metafor\}} y \texttt{\{metaviz\}} para hacer meta-análisis, así como \texttt{\{dplyr\}} para manipular y organizar la base de datos.

\begin{Shaded}
\begin{Highlighting}[]
\FunctionTok{library}\NormalTok{(metafor)}
\FunctionTok{library}\NormalTok{(metaviz)}
\FunctionTok{library}\NormalTok{(dplyr)}
\end{Highlighting}
\end{Shaded}

Una vez cargado el paquete \texttt{\{metafor\}}, ya puedo cargar la base de datos \texttt{dat.molloy2014}. En éste caso, para poder \emph{llamarla} cuando sea necesario, la asignaré a un objeto llamado \texttt{dat}.

\begin{Shaded}
\begin{Highlighting}[]
\NormalTok{dat }\OtherTok{\textless{}{-}} \FunctionTok{get}\NormalTok{(}\FunctionTok{data}\NormalTok{(dat.molloy2014))}
\end{Highlighting}
\end{Shaded}

Tras asignar la base de datos a este objeto (\texttt{dat}), podemos verla en la consola de R sencillamente usando el comando \texttt{dat}.

\begin{Shaded}
\begin{Highlighting}[]
\NormalTok{dat}
\end{Highlighting}
\end{Shaded}

\begin{ROut}{Consola de R: Output~\thetcbcounter}
                \begin{footnotesize}
                \begin{verbatim}                authors year  ni     ri controls          design   a_measure
1      Axelsson et al. 2009 109  0.187     none cross-sectional self-report
2      Axelsson et al. 2011 749  0.162     none cross-sectional self-report
3         Bruce et al. 2010  55  0.340     none     prospective       other
4   Christensen et al. 1999 107  0.320     none cross-sectional self-report
5  Christensen & Smith 1995  72  0.270     none     prospective       other
6         Cohen et al. 2004  65  0.000     none     prospective       other
7       Dobbels et al. 2005 174  0.175     none cross-sectional self-report
8        Ediger et al. 2007 326  0.050 multiple     prospective self-report
9         Insel et al. 2006  58  0.260     none     prospective       other
10       Jerant et al. 2011 771  0.010 multiple     prospective       other
11        Moran et al. 1997  56 -0.090 multiple     prospective       other
12   O'Cleirigh et al. 2007  91  0.370     none     prospective self-report
13       Penedo et al. 2003 116  0.000     none cross-sectional self-report
14        Quine et al. 2012 537  0.150     none     prospective self-report
15      Stilley et al. 2004 158  0.240     none     prospective       other
16 Wiebe & Christensen 1997  65  0.040     none     prospective       other
   c_measure meanage quality
1      other   22.00       1
2        NEO   53.59       1
3        NEO   43.36       2
4      other   41.70       1
5        NEO   46.39       2
6        NEO   41.20       2
7        NEO   52.30       1
8        NEO   41.00       3
9      other   77.00       2
10       NEO   78.60       3
11       NEO   57.20       2
12       NEO   37.90       2
13       NEO   39.20       1
14     other   69.00       2
15       NEO   46.20       3
16       NEO   56.00       1
 \end{verbatim}
                \end{footnotesize}
                \end{ROut}

Por supuesto, la salida de la consola no es la más clara, así que de aquí en adelante mostraré la mayoría de las tablas en un formato de impresión, más fácil de leer. Voy a volver a cargar la base de datos (sobreescribiendo el objeto \texttt{dat}), para organizarla un poco mejor. Primero, agregaré una nueva columna llamada \texttt{study\_id}, en la que numeraré los estudios del 1 al 16. A continuación, reorganizaré las columnas para que \texttt{study\_id} sea la primera, en vez de la última columna.

\begin{Shaded}
\begin{Highlighting}[]
\NormalTok{dat }\OtherTok{\textless{}{-}} \FunctionTok{get}\NormalTok{(}\FunctionTok{data}\NormalTok{(dat.molloy2014)) }\SpecialCharTok{\%\textgreater{}\%}
  \FunctionTok{mutate}\NormalTok{(}\AttributeTok{study\_id =} \DecValTok{1}\SpecialCharTok{:}\DecValTok{16}\NormalTok{)  }\SpecialCharTok{\%\textgreater{}\%} \CommentTok{\#agregar columna study\_id}
  \FunctionTok{select}\NormalTok{(study\_id, authors}\SpecialCharTok{:}\NormalTok{quality) }\CommentTok{\#mover study\_id como primera columna}
\end{Highlighting}
\end{Shaded}

Con esto, la base de datos tiene ahora la siguiente estructura (Tabla \ref{tab:estructuramod}):

\begin{table}[H]

\caption{\label{tab:estructuramod}Estructura de la base de datos con estudios numerados}
\centering
\resizebox{\linewidth}{!}{
\begin{tabular}[t]{rlrrrllllrr}
\toprule
study\_id & authors & year & ni & ri & controls & design & a\_measure & c\_measure & meanage & quality\\
\midrule
1 & Axelsson et al. & 2009 & 109 & 0.187 & none & cross-sectional & self-report & other & 22.00 & 1\\
2 & Axelsson et al. & 2011 & 749 & 0.162 & none & cross-sectional & self-report & NEO & 53.59 & 1\\
3 & Bruce et al. & 2010 & 55 & 0.340 & none & prospective & other & NEO & 43.36 & 2\\
4 & Christensen et al. & 1999 & 107 & 0.320 & none & cross-sectional & self-report & other & 41.70 & 1\\
5 & Christensen \& Smith & 1995 & 72 & 0.270 & none & prospective & other & NEO & 46.39 & 2\\
6 & Cohen et al. & 2004 & 65 & 0.000 & none & prospective & other & NEO & 41.20 & 2\\
7 & Dobbels et al. & 2005 & 174 & 0.175 & none & cross-sectional & self-report & NEO & 52.30 & 1\\
8 & Ediger et al. & 2007 & 326 & 0.050 & multiple & prospective & self-report & NEO & 41.00 & 3\\
9 & Insel et al. & 2006 & 58 & 0.260 & none & prospective & other & other & 77.00 & 2\\
10 & Jerant et al. & 2011 & 771 & 0.010 & multiple & prospective & other & NEO & 78.60 & 3\\
11 & Moran et al. & 1997 & 56 & -0.090 & multiple & prospective & other & NEO & 57.20 & 2\\
12 & O'Cleirigh et al. & 2007 & 91 & 0.370 & none & prospective & self-report & NEO & 37.90 & 2\\
13 & Penedo et al. & 2003 & 116 & 0.000 & none & cross-sectional & self-report & NEO & 39.20 & 1\\
14 & Quine et al. & 2012 & 537 & 0.150 & none & prospective & self-report & other & 69.00 & 2\\
15 & Stilley et al. & 2004 & 158 & 0.240 & none & prospective & other & NEO & 46.20 & 3\\
16 & Wiebe \& Christensen & 1997 & 65 & 0.040 & none & prospective & other & NEO & 56.00 & 1\\
\bottomrule
\multicolumn{11}{l}{\rule{0pt}{1em}\textit{Nota:} Datos tomados de Molloy et al. (2013).}\\
\end{tabular}}
\end{table}

Por supuesto, la columna \texttt{authors} tiene los autores de cada estudio a meta-analizar, y la columna \texttt{year} el año de publicación.

La columna \texttt{ri} contiene los coeficientes de correlación de Pearson (la columna \texttt{ni} contiene los tamaños de muestra de cada estudio).

Adicionalmente, en este ejemplo tenemos una serie de moderadores:

\begin{itemize}
\item
  \texttt{controls}: cantidad de variables controladas (ninguna o múltiples)
\item
  \texttt{design}: si se utilizó un diseño transversal o prospectivo
\item
  \texttt{a\_measure}: tipo de medida de adherencia (autoinforme u otro)
\item
  \texttt{c\_measure}: tipo de medida de diligencia (NEO u otra)
\item
  \texttt{meanage}: edad promedio de la muestra
\item
  \texttt{quality}: calidad metodológica
\end{itemize}

\hypertarget{transformaciuxf3n-de-r-de-pearson-a-z-de-fisher}{%
\section{\texorpdfstring{Transformación de \emph{r} de Pearson a \emph{z} de Fisher}{Transformación de r de Pearson a z de Fisher}}\label{transformaciuxf3n-de-r-de-pearson-a-z-de-fisher}}

Dado que los coeficientes de Pearson (columna \texttt{ri}) no tienen una distribución normal, esto podría llevar a calcular varianzas incorrectas, especialmente cuando se trata de correlaciones con tamaños de muestra pequeños. Por esto, vamos a transformar los coeficientes \emph{r} de Pearson a \emph{z} de Fisher, que no tienen este problema. Usaré la función \texttt{escalc} del paquete \texttt{metafor}.

\begin{Shaded}
\begin{Highlighting}[]
\NormalTok{dat }\OtherTok{\textless{}{-}} \FunctionTok{escalc}\NormalTok{(}\AttributeTok{measure =} \StringTok{"ZCOR"}\NormalTok{, }
              \AttributeTok{ri =}\NormalTok{ ri, }
              \AttributeTok{ni =}\NormalTok{ ni,}
              \AttributeTok{data =}\NormalTok{ dat)}
\end{Highlighting}
\end{Shaded}

Esto ha creado dos nuevas variables en nuestra tabla: \texttt{yi}, que es el tamaño de efecto, y \texttt{vi} que es la varianza (Tabla \ref{tab:estructuramod2}).

\begin{table}[H]

\caption{\label{tab:estructuramod2}Estructura de la base de datos, con transformación de los r de Pearon a z de Fisher}
\centering
\resizebox{\linewidth}{!}{
\begin{tabular}[t]{rlrrrllllrr>{}r>{}r}
\toprule
study\_id & authors & year & ni & ri & controls & design & a\_measure & c\_measure & meanage & quality & yi & vi\\
\midrule
1 & Axelsson et al. & 2009 & 109 & 0.187 & none & cross-sectional & self-report & other & 22.00 & 1 & \cellcolor[HTML]{f68212}{0.1892266} & \cellcolor[HTML]{f68212}{0.0094340}\\
2 & Axelsson et al. & 2011 & 749 & 0.162 & none & cross-sectional & self-report & NEO & 53.59 & 1 & \cellcolor[HTML]{f68212}{0.1634399} & \cellcolor[HTML]{f68212}{0.0013405}\\
3 & Bruce et al. & 2010 & 55 & 0.340 & none & prospective & other & NEO & 43.36 & 2 & \cellcolor[HTML]{f68212}{0.3540925} & \cellcolor[HTML]{f68212}{0.0192308}\\
4 & Christensen et al. & 1999 & 107 & 0.320 & none & cross-sectional & self-report & other & 41.70 & 1 & \cellcolor[HTML]{f68212}{0.3316471} & \cellcolor[HTML]{f68212}{0.0096154}\\
5 & Christensen \& Smith & 1995 & 72 & 0.270 & none & prospective & other & NEO & 46.39 & 2 & \cellcolor[HTML]{f68212}{0.2768638} & \cellcolor[HTML]{f68212}{0.0144928}\\
6 & Cohen et al. & 2004 & 65 & 0.000 & none & prospective & other & NEO & 41.20 & 2 & \cellcolor[HTML]{f68212}{0.0000000} & \cellcolor[HTML]{f68212}{0.0161290}\\
7 & Dobbels et al. & 2005 & 174 & 0.175 & none & cross-sectional & self-report & NEO & 52.30 & 1 & \cellcolor[HTML]{f68212}{0.1768200} & \cellcolor[HTML]{f68212}{0.0058480}\\
8 & Ediger et al. & 2007 & 326 & 0.050 & multiple & prospective & self-report & NEO & 41.00 & 3 & \cellcolor[HTML]{f68212}{0.0500417} & \cellcolor[HTML]{f68212}{0.0030960}\\
9 & Insel et al. & 2006 & 58 & 0.260 & none & prospective & other & other & 77.00 & 2 & \cellcolor[HTML]{f68212}{0.2661084} & \cellcolor[HTML]{f68212}{0.0181818}\\
10 & Jerant et al. & 2011 & 771 & 0.010 & multiple & prospective & other & NEO & 78.60 & 3 & \cellcolor[HTML]{f68212}{0.0100003} & \cellcolor[HTML]{f68212}{0.0013021}\\
11 & Moran et al. & 1997 & 56 & -0.090 & multiple & prospective & other & NEO & 57.20 & 2 & \cellcolor[HTML]{f68212}{-0.0902442} & \cellcolor[HTML]{f68212}{0.0188679}\\
12 & O'Cleirigh et al. & 2007 & 91 & 0.370 & none & prospective & self-report & NEO & 37.90 & 2 & \cellcolor[HTML]{f68212}{0.3884231} & \cellcolor[HTML]{f68212}{0.0113636}\\
13 & Penedo et al. & 2003 & 116 & 0.000 & none & cross-sectional & self-report & NEO & 39.20 & 1 & \cellcolor[HTML]{f68212}{0.0000000} & \cellcolor[HTML]{f68212}{0.0088496}\\
14 & Quine et al. & 2012 & 537 & 0.150 & none & prospective & self-report & other & 69.00 & 2 & \cellcolor[HTML]{f68212}{0.1511404} & \cellcolor[HTML]{f68212}{0.0018727}\\
15 & Stilley et al. & 2004 & 158 & 0.240 & none & prospective & other & NEO & 46.20 & 3 & \cellcolor[HTML]{f68212}{0.2447741} & \cellcolor[HTML]{f68212}{0.0064516}\\
16 & Wiebe \& Christensen & 1997 & 65 & 0.040 & none & prospective & other & NEO & 56.00 & 1 & \cellcolor[HTML]{f68212}{0.0400214} & \cellcolor[HTML]{f68212}{0.0161290}\\
\bottomrule
\multicolumn{13}{l}{\rule{0pt}{1em}\textit{Nota:} Las nuevas columnas creadas usando la función \texttt{escalc} 
           (\texttt{yi} como tamaño de efecto y \texttt{vi} como varianza) están 
           resaltadas en naranja}\\
\end{tabular}}
\end{table}

\hypertarget{meta-cor}{%
\section{Hacer el meta-análisis}\label{meta-cor}}

Para hacer el meta-análisis, usaremos la función \texttt{rma} del paquete \texttt{metafor}, para el que tenemos que especificar los tamaños de efecto (\texttt{yi}) y varianzas (\texttt{vi}) de los estudios a meta-analizar. En este caso, las columnas donde tenemos estos valores, tienen los mismos nombres (\texttt{yi}, \texttt{vi}). Asignaré los resultados del meta-análisis a un objeto llamado \texttt{res}.

\begin{Shaded}
\begin{Highlighting}[]
\NormalTok{res }\OtherTok{\textless{}{-}} \FunctionTok{rma}\NormalTok{(}\AttributeTok{yi =}\NormalTok{ yi, }\AttributeTok{vi =}\NormalTok{ vi, }\AttributeTok{data =}\NormalTok{ dat)}
\end{Highlighting}
\end{Shaded}

Los resultados, son los siguientes:

\begin{Shaded}
\begin{Highlighting}[]
\NormalTok{res}
\end{Highlighting}
\end{Shaded}

\begin{ROut}{Consola de R: Output~\thetcbcounter}
                \begin{footnotesize}
                \begin{verbatim} 
Random-Effects Model (k = 16; tau^2 estimator: REML)

tau^2 (estimated amount of total heterogeneity): 0.0081 (SE = 0.0055)
tau (square root of estimated tau^2 value):      0.0901
I^2 (total heterogeneity / total variability):   61.73%
H^2 (total variability / sampling variability):  2.61

Test for Heterogeneity:
Q(df = 15) = 38.1595, p-val = 0.0009

Model Results:

estimate      se    zval    pval   ci.lb   ci.ub 
  0.1499  0.0316  4.7501  <.0001  0.0881  0.2118  *** 

---
Signif. codes:  0 '***' 0.001 '**' 0.01 '*' 0.05 '.' 0.1 ' ' 1
 \end{verbatim}
                \end{footnotesize}
                \end{ROut}

Primero, nos confirma que ajustamos un modelo con efectos aleatorios (\texttt{Random-Effects\ Model}), a partir de 16 estudios (\texttt{k\ =\ 16}), y que para estimar \(\tau^2\) (tau cuadrado) usamos el método de \textbf{máxima verosimilitud restringida}\footnote{Hay varios métodos disponibles como estimador, además de \textbf{máxima verosimilitud restringida} (REML). Sin embargo, si tienes dudas, REML es una buena opción. Cada método tiene ventajas y desventajas que, si tienes interés en mirar, están descritas en la \href{https://www.rdocumentation.org/packages/metafor/versions/2.4-0/topics/rma.uni}{documentación} de la función \texttt{rma}.} (\texttt{tau\^{}2\ estimator:\ REML}), que se designa como \emph{REML} por sus siglas en inglés .

Posteriormente, nos provee los valores de una serie de estimadores de heterogeneidad o varianza:

\begin{itemize}
\item
  \(\tau^2\): \texttt{tau\^{}2\ (estimated\ amount\ of\ total\ heterogeneity):\ 0.0081\ (SE\ =\ 0.0055)}
\item
  \(\tau\): \texttt{tau\ (square\ root\ of\ estimated\ tau\^{}2\ value):\ 0.0901}
\item
  \(I^2\): \texttt{I\^{}2\ (total\ heterogeneity\ /\ total\ variability):\ 61.73\%}, y
\item
  \(H^2\): \texttt{H\^{}2\ (total\ variability\ /\ sampling\ variability):\ \ 2.61}
\end{itemize}

La tercera parte, reporta una prueba de heterogeneidad, usando el estadístico \(Q\):

\begin{itemize}
\item
  \texttt{Test\ for\ Heterogeneity:}

  \texttt{Q(df\ =\ 15)\ =\ 38.1595,\ p-val\ =\ 0.0009}
\end{itemize}

De todos estos, los más comúnmente reportados son \(\tau^2\), \(\tau\), \(I^2\) y \(Q\). Cada una de estas medidas tiene ventajas y desventajas, por lo cual tiene sentido reportarlas todas.

\begin{itemize}
\item
  \textbf{\(I^2\):} tiene la ventaja de ser sencillo de interpretar, pues hay criterios generales para heterogeneidad baja, moderada y alta (típicamente 25\%, 50\%, and 75\%, respectivamente). Sin embargo, es muy sensible a los tamaños de muestra de los estudios meta-analizados (por ejemplo, si en tu meta-análisis hay estudios con tamaños de muestra muy grandes, esto va a sesgar tu \(I^2\)).
\item
  \textbf{\(Q\):} aunque no es sensible al tamaño de muestra, es sensible al número de estudios meta-analizados. Tiene la ventaja de ser un test de hipótesis, y como tal, puede ser interpretado a partir de su valor \emph{p}.
\item
  \textbf{\(\tau^2\):} no tiene problemas de sensibilidad a los tamaños de muestra o número de estudios meta-analizados, pero es más difícil de interpretar. \(\tau^2\) es una estimación de la varianza de los tamaños de los efectos reales entre los estudios meta-analizados. Se usa, principalmente, para asignar pesos a cada estudio. Para más información, ver Borenstein et al. (\protect\hyperlink{ref-borensteinIdentifyingQuantifyingHeterogeneity2009}{2009}).
\end{itemize}

En nuestro caso, el estadístico \(Q\) sugiere que hay una heterogeneidad significativa en los estudios meta-analizados (\(p\) = 0.0009). \(I^2\), sugiere una heterogeneidad moderada, lo que quiere decir que más de la mitad (61.73\%) de la varianza se estima que se deriva de diferencias en los tamaños de efecto.

Por último, tenemos los resultados del modelo de meta-análisis (\texttt{Model\ results}). Nos provee un estimado de la asociación positiva entre diligencia y adherencia a la medicación (0.1499 ± 0.0316), lo que equivale a un valor \emph{z} de 4.7501 , y sugiere que esa asociación es significativa (\(p\) \textless{} .0001). Así mismo, nos provee los límites inferior (0.0881) y superior (0.2118) de los intervalos de confianza.

\hypertarget{muxe1s-informaciuxf3n-sobre-heterogeneidad}{%
\subsection{Más información sobre heterogeneidad}\label{muxe1s-informaciuxf3n-sobre-heterogeneidad}}

Además de reportar los estadísticos \(\tau^2\), \(\tau\), \(I^2\) y \(Q\), podemos fácilmente calcular los intervalos de confianza para \(\tau^2\), \(\tau\), e \(I^2\) con la función \texttt{confint}, que también pueden ser reportado junto a estos estadísticos.

\begin{Shaded}
\begin{Highlighting}[]
\FunctionTok{confint}\NormalTok{(res)}
\end{Highlighting}
\end{Shaded}

\begin{ROut}{Consola de R: Output~\thetcbcounter}
                \begin{footnotesize}
                \begin{verbatim} 
       estimate   ci.lb   ci.ub 
tau^2    0.0081  0.0017  0.0378 
tau      0.0901  0.0412  0.1944 
I^2(%)  61.7324 25.2799 88.2545 
H^2      2.6132  1.3383  8.5139 
 \end{verbatim}
                \end{footnotesize}
                \end{ROut}

Para el \(\tau^2\), el hecho de que los intervalos de confianza no crucen el 0 (en nuestro caso 0.0017 \emph{---} 0.0378), sugiere que de hecho también que hay heterogeneidad entre los estudios que meta-analizamos.

\hypertarget{diagnuxf3stico-de-influencia}{%
\subsection{Diagnóstico de influencia}\label{diagnuxf3stico-de-influencia}}

Otro aspecto importante de un meta-análisis, es determinar si alguno(s) de los estudios meta-analizados es(son) particularmente influyente(s) en nuestro resultado\footnote{Por ejemplo, si estuviésemos meta-analizando 20 estudios, de los cuales 19 tienen un \emph{n} de 100, pero el otro tiene un \emph{n} de 10.000, éste último tendrá una influencia enorme en nuestro resultado. Sería preocupante que tu meta-análisis sea dependiente de un único estudio.}. Para esto, podemos usar la función \texttt{influence}, cuyo resultado en este caso asignaré a un objeto llamado \texttt{inf.res}.

\begin{Shaded}
\begin{Highlighting}[]
\NormalTok{inf.res }\OtherTok{\textless{}{-}} \FunctionTok{influence}\NormalTok{(res)}
\end{Highlighting}
\end{Shaded}

Ya que lo asigné a un objeto (\texttt{inf.res}), para ver el resultado, debo correr este objeto para ver su resultado.

\begin{Shaded}
\begin{Highlighting}[]
\NormalTok{inf.res}
\end{Highlighting}
\end{Shaded}

\begin{ROut}{Consola de R: Output~\thetcbcounter}
                \begin{footnotesize}
                \begin{verbatim} 
   rstudent  dffits cook.d  cov.r tau2.del  QE.del    hat  weight    dfbs inf 
1    0.2918  0.0485 0.0025 1.1331   0.0091 37.7109 0.0568  5.6776  0.0481     
2    0.1196 -0.0031 0.0000 1.2595   0.0100 36.7672 0.1054 10.5396 -0.0032     
3    1.2740  0.2595 0.0660 0.9942   0.0075 35.3930 0.0364  3.6432  0.2623     
4    1.4711  0.3946 0.1439 0.9544   0.0068 33.5886 0.0562  5.6195  0.3994     
5    0.8622  0.1838 0.0339 1.0505   0.0082 36.5396 0.0441  4.4069  0.1837     
6   -0.9795 -0.2121 0.0455 1.0639   0.0084 37.1703 0.0411  4.1094 -0.2112     
7    0.2177  0.0296 0.0010 1.1740   0.0094 37.6797 0.0714  7.1362  0.0296     
8   -0.9774 -0.3120 0.1001 1.1215   0.0084 36.1484 0.0889  8.8886 -0.3128     
9    0.7264  0.1392 0.0195 1.0561   0.0083 37.0495 0.0379  3.7886  0.1387     
10  -1.8667 -0.5861 0.2198 0.8502   0.0047 25.0661 0.1058 10.5826 -0.5430     
11  -1.4985 -0.2771 0.0756 1.0073   0.0077 35.6617 0.0369  3.6922 -0.2791     
12   1.8776  0.4918 0.2148 0.8819   0.0059 31.9021 0.0511  5.1150  0.5059     
13  -1.1892 -0.2939 0.0859 1.0550   0.0080 36.3291 0.0587  5.8732 -0.2941     
14  -0.0020 -0.0423 0.0021 1.2524   0.0100 37.7339 0.0998  9.9778 -0.0434     
15   0.8066  0.2126 0.0459 1.0907   0.0083 35.8385 0.0684  6.8403  0.2125     
16  -0.7160 -0.1656 0.0280 1.0853   0.0087 37.7017 0.0411  4.1094 -0.1642     
 \end{verbatim}
                \end{footnotesize}
                \end{ROut}

Esto me muestra gran cantidad de información de cada estudio (en este caso, una tabla sin formato). Sin embargo, lo más importante ahora es mirar la última columna, llamada \texttt{inf}. Si ahí aparecieran asteriscos (que no es nuestro caso), sugeriría que ese estudio es particularmente influyente.

Por último, podemos también ver ésta información que tenemos guardada en el objeto \texttt{inf.res}, de manera gráfica, usando la función \texttt{plot} (Fig. \ref{fig:infplot}).

\begin{Shaded}
\begin{Highlighting}[]
\FunctionTok{plot}\NormalTok{(inf.res)}
\end{Highlighting}
\end{Shaded}

\begin{figure}
\centering
\includegraphics{Meta-analysis_files/figure-latex/infplot-1.pdf}
\caption{\label{fig:infplot}Diagnóstico de influencia. Estudios particularmente influyentes serían representados con un punto rojo. En este caso, no hay ningún estudio que se considere demasiado influyente, por lo éste análisis sugiere que podemos estar tranquilos con nuestro meta-análisis.}
\end{figure}

\hypertarget{forest-plot-diagrama-de-bosque}{%
\subsection{\texorpdfstring{\emph{Forest plot} (diagrama de bosque)}{Forest plot (diagrama de bosque)}}\label{forest-plot-diagrama-de-bosque}}

Para hacer un diagrama de bosque (\emph{forest plot}) con \href{https://www.metafor-project.org/doku.php}{\texttt{metafor}} resumiendo nuestro meta-análisis, solo tenemos que usar la función \texttt{forest}, usando como argumento el objeto al que asignamos los resultados de nuestro meta-análisis (\texttt{res}; esto produce la Fig. \ref{fig:for-plot1}).

\begin{Shaded}
\begin{Highlighting}[]
\FunctionTok{forest}\NormalTok{(res)}
\end{Highlighting}
\end{Shaded}

\begin{figure}
\centering
\includegraphics{Meta-analysis_files/figure-latex/for-plot1-1.pdf}
\caption{\label{fig:for-plot1}Forest plot básico de \href{https://www.metafor-project.org/doku.php}{metafor}. Para cada estudio meta-analizado, tenemos el efecto (correlación, en este caso en valores \emph{z} de Fisher), así como sus intervalos de confianza entre paréntesis cuadrados. Esta misma información está representada gráficamente, con los cuadrados representando el efecto de cada estudio así como sus intervalos de confianza, y el tamaño de muestra (representado por el tamaño del cuadrado). Bajo estos resultados, tenemos nuestro meta-análisis, con el mismo formato en texto, pero representando el efecto y sus intervalos de confianza con un diamante.}
\end{figure}

Como se puede ver en las Figuras \ref{fig:for-plot1}, \ref{fig:for-plot2} y \ref{fig:for-plot3b} (que son versiones del mismo \emph{forest plot}), no es una sorpresa que el análisis nos sugiera bastante heterogeneidad; las correlaciones encontradas entre los diferentes estudios varían mucho (están entre -0.09 y 0.37), y aunque son positivas en la mayoría de los casos (en algunos claramente positivas), en algunos son prácticamente 0 o incluso negativas.

Para una versión más completa y anotada, también usando el \emph{plot} básico de \href{https://www.metafor-project.org/doku.php}{\texttt{metafor}}, pero agregando encabezados de cada columna en español, nombres de los estudios meta-analizados\footnote{En este caso, y dado que tenemos la lista de autores y años de publicación en columnas separadas, pegando las columnas \texttt{authors} y \texttt{year} separadas por una coma y un espacio: \texttt{paste(dat\$authors,\ dat\$year,\ sep\ =\ ",\ ")} como argumento \texttt{slab}.} así como una columna con los pesos dados a cada estudio, y detalles del modelo final\footnote{Estas opciones están explicadas \href{https://search.r-project.org/CRAN/refmans/metafor/html/forest.rma.html}{acá}.}, podemos hacer algo como esto:

\begin{Shaded}
\begin{Highlighting}[]
\CommentTok{\# forest plot con anotaciones adicionales}
\FunctionTok{forest}\NormalTok{(res, }\AttributeTok{cex =} \FloatTok{0.75}\NormalTok{, }\AttributeTok{xlim =} \FunctionTok{c}\NormalTok{(}\SpecialCharTok{{-}}\FloatTok{1.6}\NormalTok{, }\FloatTok{1.6}\NormalTok{),}
       \AttributeTok{slab =} \FunctionTok{paste}\NormalTok{(dat}\SpecialCharTok{$}\NormalTok{authors, dat}\SpecialCharTok{$}\NormalTok{year, }\AttributeTok{sep =} \StringTok{", "}\NormalTok{),}
       \AttributeTok{showweights =} \ConstantTok{TRUE}\NormalTok{,}
       \AttributeTok{xlab =} \StringTok{"Coeficiente de correlación transformado en z de Fisher"}\NormalTok{,}
       \AttributeTok{digits =} \FunctionTok{c}\NormalTok{(}\DecValTok{2}\NormalTok{,3L),}
       \AttributeTok{mlab =} \FunctionTok{bquote}\NormalTok{(}\FunctionTok{paste}\NormalTok{(}\StringTok{"Modelo EA: Q("}\NormalTok{, .(res}\SpecialCharTok{$}\NormalTok{k }\SpecialCharTok{{-}}\NormalTok{ res}\SpecialCharTok{$}\NormalTok{p), }\StringTok{") = "}\NormalTok{,}
\NormalTok{     .(}\FunctionTok{formatC}\NormalTok{(res}\SpecialCharTok{$}\NormalTok{QE, }\AttributeTok{digits=}\DecValTok{2}\NormalTok{, }\AttributeTok{format=}\StringTok{"f"}\NormalTok{)),}
     \StringTok{", p "}\NormalTok{, .(scales}\SpecialCharTok{::}\FunctionTok{pvalue}\NormalTok{(res}\SpecialCharTok{$}\NormalTok{pval)), }\StringTok{"; "}\NormalTok{, I}\SpecialCharTok{\^{}}\DecValTok{2}\NormalTok{, }\StringTok{" = "}\NormalTok{,}
\NormalTok{     .(}\FunctionTok{formatC}\NormalTok{(res}\SpecialCharTok{$}\NormalTok{I2, }\AttributeTok{digits=}\DecValTok{1}\NormalTok{, }\AttributeTok{format=}\StringTok{"f"}\NormalTok{)), }\StringTok{"\%"}\NormalTok{)))}
\CommentTok{\# agregar encabezados a las columnas (valores de X y Y deben ser ajustados)}
\NormalTok{op }\OtherTok{\textless{}{-}} \FunctionTok{par}\NormalTok{(}\AttributeTok{cex =} \FloatTok{0.8}\NormalTok{, }\AttributeTok{font=}\DecValTok{2}\NormalTok{) }
\FunctionTok{text}\NormalTok{(}\AttributeTok{x =} \SpecialCharTok{{-}}\FloatTok{1.6}\NormalTok{, }\AttributeTok{y =} \DecValTok{18}\NormalTok{, }\AttributeTok{labels =} \StringTok{"Autor(es), Año"}\NormalTok{, }\AttributeTok{pos =} \DecValTok{4}\NormalTok{)}
\FunctionTok{text}\NormalTok{(}\AttributeTok{x =} \DecValTok{0}\NormalTok{, }\AttributeTok{y =} \DecValTok{18}\NormalTok{, }\AttributeTok{labels =} \StringTok{"Efecto e IC"}\NormalTok{, }\AttributeTok{pos =} \DecValTok{4}\NormalTok{)}
\FunctionTok{text}\NormalTok{(}\AttributeTok{x =} \DecValTok{1}\NormalTok{, }\AttributeTok{y =} \DecValTok{18}\NormalTok{, }\AttributeTok{labels =} \StringTok{"Peso"}\NormalTok{, }\AttributeTok{pos =} \DecValTok{2}\NormalTok{)}
\FunctionTok{text}\NormalTok{(}\AttributeTok{x =} \FloatTok{1.6}\NormalTok{, }\AttributeTok{y =} \DecValTok{18}\NormalTok{, }\AttributeTok{labels =} \StringTok{"Corr. [95\% IC]"}\NormalTok{, }\AttributeTok{pos =} \DecValTok{2}\NormalTok{)}
\end{Highlighting}
\end{Shaded}

\begin{figure}
\centering
\includegraphics{Meta-analysis_files/figure-latex/for-plot2-1.pdf}
\caption{\label{fig:for-plot2}Forest plot anotado, creado con \href{https://www.metafor-project.org/doku.php}{metafor}. En esta versión agregué algunos encabezados en español, así como estadísticos generales del modelo de meta-análisis. Modelo EA se refiere al modelo meta-analizado, de efectos aleatorios.}
\end{figure}

O, para una incluso más sofisticada, se puede usar la función \href{https://cran.r-project.org/web/packages/metaviz/vignettes/metaviz.html\#creating-forest-plots-with-function-viz_forest}{\texttt{viz\_forest}} del paquete \href{https://cran.r-project.org/web/packages/metaviz/vignettes/metaviz.html}{\texttt{metaviz}}.

\begin{Shaded}
\begin{Highlighting}[]
\CommentTok{\# A. Variante "classic" (no tiene que ser definida, pues es la opción por defecto)}
\FunctionTok{viz\_forest}\NormalTok{(res, }
           \AttributeTok{study\_labels =} \FunctionTok{paste}\NormalTok{(dat}\SpecialCharTok{$}\NormalTok{authors, dat}\SpecialCharTok{$}\NormalTok{year, }\AttributeTok{sep =} \StringTok{", "}\NormalTok{),}
           \AttributeTok{xlab =} \StringTok{"Correlación"}\NormalTok{, }
           \AttributeTok{annotate\_CI =} \ConstantTok{TRUE}\NormalTok{,}
           \AttributeTok{summary\_label =} \StringTok{"Resumen"}\NormalTok{,}
           \AttributeTok{text\_size =} \FloatTok{2.6}\NormalTok{,}
           \AttributeTok{x\_trans\_function =}\NormalTok{ tanh)}

\CommentTok{\# B. Variante "thick"}
\FunctionTok{viz\_forest}\NormalTok{(res, }
           \AttributeTok{study\_labels =} \FunctionTok{paste}\NormalTok{(dat}\SpecialCharTok{$}\NormalTok{authors, dat}\SpecialCharTok{$}\NormalTok{year, }\AttributeTok{sep =} \StringTok{", "}\NormalTok{),}
           \AttributeTok{xlab =} \StringTok{"Correlación"}\NormalTok{, }
           \AttributeTok{variant =} \StringTok{"thick"}\NormalTok{,}
           \AttributeTok{col =} \StringTok{"Greens"}\NormalTok{,}
           \AttributeTok{annotate\_CI =} \ConstantTok{TRUE}\NormalTok{,}
           \AttributeTok{summary\_label =} \StringTok{"Resumen"}\NormalTok{,}
           \AttributeTok{text\_size =} \FloatTok{2.6}\NormalTok{,}
           \AttributeTok{x\_trans\_function =}\NormalTok{ tanh)}

\CommentTok{\# C. Variante "rain"}
\FunctionTok{viz\_forest}\NormalTok{(res, }
           \AttributeTok{study\_labels =} \FunctionTok{paste}\NormalTok{(dat}\SpecialCharTok{$}\NormalTok{authors, dat}\SpecialCharTok{$}\NormalTok{year, }\AttributeTok{sep =} \StringTok{", "}\NormalTok{),}
           \AttributeTok{xlab =} \StringTok{"Correlación"}\NormalTok{, }
           \AttributeTok{variant =} \StringTok{"rain"}\NormalTok{,}
           \AttributeTok{col =} \StringTok{"Oranges"}\NormalTok{,}
           \AttributeTok{annotate\_CI =} \ConstantTok{TRUE}\NormalTok{,}
           \AttributeTok{summary\_label =} \StringTok{"Resumen"}\NormalTok{,}
           \AttributeTok{text\_size =} \FloatTok{2.6}\NormalTok{,}
           \AttributeTok{x\_trans\_function =}\NormalTok{ tanh)}
\end{Highlighting}
\end{Shaded}

Con el código anterior genero las siguientes tres versiones del mismo \emph{forest plot} usando diferentes variantes y escalas de colores, y transformando de vuelta los coeficientes de \emph{z} de Fisher a \emph{r} de Pearson.

Por supuesto, es cuestión de gusto cuál usar.

\includegraphics{Meta-analysis_files/figure-latex/for-plot3-1.pdf}

\begin{figure}
\centering
\includegraphics{Meta-analysis_files/figure-latex/for-plot3b-1.pdf}
\caption{\label{fig:for-plot3b}Variantes de forest plots creados con \href{https://cran.r-project.org/web/packages/metaviz/vignettes/metaviz.html}{metaviz}. \textbf{A.} Variante clásica (opción por defecto). \textbf{B.} Variante ``thick'' y escala de colores ``Greens''. \textbf{C.} Variante ``rain'' y escala de colores ``Oranges''.}
\end{figure}

\hypertarget{funnel-plot-diagrama-de-embudo-y-sesgo-de-estudios-pequeuxf1os}{%
\subsection{\texorpdfstring{\emph{Funnel plot} (diagrama de embudo) y sesgo de estudios pequeños}{Funnel plot (diagrama de embudo) y sesgo de estudios pequeños}}\label{funnel-plot-diagrama-de-embudo-y-sesgo-de-estudios-pequeuxf1os}}

En este punto, es en donde más errores se cometen. Las pruebas más comunes para evaluar sesgos de publicación, son la evaluación de la asimatría en el \emph{funnel plot} (diagrama de embudo), y la regresión (o test) de Egger (\protect\hyperlink{ref-eggerBiasMetaanalysisDetected1997}{Egger et al., 1997}).

El principal error que la mayoría de los investigadores (meta-analistas) cometen, es que simplemente basándose en éstos métodos, concluyen que un meta-análisis tiene (o no) riesgo de sufrir de un sesgo de publicación. Sin embargo, estos métodos, no son pruebas exclusivas de sesgo de publicación, sino de sesgo de estudios de tamaño muestral pequeño (ver e.g. \protect\hyperlink{ref-schwarzerSmallStudyEffectsMetaAnalysis2015}{Schwarzer et al., 2015b}), que pueden incluir sesgo de publicación, pero no se centran exclusivamente en éste.

A pesar de esto, tanto la regresión de Egger como el \emph{funnel plot}, son intersantes dado que el sesgo de estudios pequeños es importante.

\hypertarget{funnel-plot}{%
\subsubsection{\texorpdfstring{\emph{Funnel plot}}{Funnel plot}}\label{funnel-plot}}

Para crear un \emph{funnel plot} con \href{https://www.metafor-project.org/doku.php}{\texttt{metafor}}, de nuestro meta-análisis, solo tenemos que usar la función \texttt{funnel}, usando como argumento el objeto al que asignamos los resultados de nuestro meta-análisis (\texttt{res}). Con esto, he generado la Figura \ref{fig:funnel-plot1}.

\begin{Shaded}
\begin{Highlighting}[]
\FunctionTok{funnel}\NormalTok{(res)}
\end{Highlighting}
\end{Shaded}

\begin{figure}
\centering
\includegraphics{Meta-analysis_files/figure-latex/funnel-plot1-1.pdf}
\caption{\label{fig:funnel-plot1}Funnel plot básico de \href{https://www.metafor-project.org/doku.php}{metafor}. Para cada estudio meta-analizado, tenemos el efecto (correlación, en este caso en valores \emph{z} de Fisher) en el eje \emph{X}, así como su error estándar en el eje \emph{Y}. La línea punteada vertical representa el efecto meta-analizado que hemos encontrado, así que podemos ver los estudios que encontraron un efecto mayor (derecha de la línea punteada) o menor (izquierda) de éste. A primera vista no parece haber mucha asimetría, pero es importante tener en cuenta que es un análisis muy subjetivo.}
\end{figure}

O, si queremos cambiar los títulos de los ejes, por ejemplo escribiéndolos en español, podemos hacerlo agregando los argumentos \texttt{xlab} (para el eje \(X\)) y/o \texttt{ylab} (para el eje \(Y\)), como se ve en la Figura \ref{fig:funnel-plot1a}.

\begin{Shaded}
\begin{Highlighting}[]
\FunctionTok{funnel}\NormalTok{(res, }
       \AttributeTok{xlab =} \StringTok{"Coeficiente de correlación transformado en z de Fisher"}\NormalTok{,}
       \AttributeTok{ylab =} \StringTok{"Error estándar"}\NormalTok{)}
\end{Highlighting}
\end{Shaded}

\begin{figure}
\centering
\includegraphics{Meta-analysis_files/figure-latex/funnel-plot1a-1.pdf}
\caption{\label{fig:funnel-plot1a}Funnel plot básico de \href{https://www.metafor-project.org/doku.php}{metafor}, con tpitulos de ejes en español. Para cada estudio meta-analizado, tenemos el efecto (correlación, en este caso en valores \emph{z} de Fisher) en el eje \emph{X}, así como su error estándar en el eje \emph{Y}. La línea punteada vertical representa el efecto meta-analizado que hemos encontrado, así que podemos ver los estudios que encontraron un efecto mayor (derecha de la línea punteada) o menor (izquierda) de éste.}
\end{figure}

De nuevo, se puede usar el paquete \href{https://cran.r-project.org/web/packages/metaviz/vignettes/metaviz.html}{\texttt{metaviz}}, usando la función \href{https://cran.r-project.org/web/packages/metaviz/vignettes/metaviz.html\#creating-funnel-plots-with-viz_funnel}{\texttt{viz\_funnel}}. Hay muchas opciones, pero como ejemplo, usaré la versión por defecto, agregando solo la línea de la regresión de Egger (\texttt{egger\ =\ TRUE}; ver sección \ref{reg-egger}, más adelante), transformando los tamaños de efecto de regreso a \(r\) de Pearson (\texttt{x\_trans\_function\ =\ tanh}), y con los títulos de los ejes en español (Fig. \ref{fig:funnel-plot2}).

\begin{Shaded}
\begin{Highlighting}[]
\FunctionTok{viz\_funnel}\NormalTok{(res, }
           \AttributeTok{egger =} \ConstantTok{TRUE}\NormalTok{,}
           \AttributeTok{x\_trans\_function =}\NormalTok{ tanh,}
           \AttributeTok{ylab =} \StringTok{"Error estándar"}\NormalTok{,}
           \AttributeTok{xlab =} \StringTok{"Coeficiente de correlación"}\NormalTok{)}
\end{Highlighting}
\end{Shaded}

\begin{figure}
\centering
\includegraphics{Meta-analysis_files/figure-latex/funnel-plot2-1.pdf}
\caption{\label{fig:funnel-plot2}Funnel plot creado con \href{https://cran.r-project.org/web/packages/metaviz/vignettes/metaviz.html}{metaviz}. En azul, se representa el área donde estudios, según su error (y su tamaño de muestra), tendrían un efecto significativo al 5\% (i.e.~\(p\) \textgreater{} 0.05), y fuera de ésta, donde tendrían un efecto significativo al 1\% (i.e.~\(p\) \textgreater{} 0.01). La línea negra vertical representa el efecto meta-analizado, y el triángulo a partir de su inicio, el área donde se ubican los estudios que no se diferencian significativamente del resultado del meta-análisis. La línea roja punteada, representa la regresión de Egger.}
\end{figure}

Alternativamente, el paquete \href{https://cran.r-project.org/web/packages/metaviz/vignettes/metaviz.html}{\texttt{metaviz}} tiene la función \href{https://cran.r-project.org/web/packages/metaviz/vignettes/metaviz.html\#sunset-power-enhanced-funnel-plots}{\texttt{viz\_sunset}}, que permite además mostrar el poder estadístico (o potencia) de los estudios meta-analizados para detectar un efecto de interés mediante una prueba de Wald de dos colas. De ser necesario, para entender las bases del poder estadístico, recomiendo ver \href{https://youtube.com/playlist?list=PLHk7UNt35ccVdyHqnQ6oXVYA6JBNFrE1x}{esta serie de videos} (\protect\hyperlink{ref-leongomezPoderRvid2020}{Leongómez, 2020b}) y/o, para mayor profundidad, leer \href{https://doi.org/10.5281/zenodo.3988776}{esta guía} (\protect\hyperlink{ref-leongomezAnalisisPoderEstadistico2020}{Leongómez, 2020a}) que publiqué anteriormente.

A continuación, muestro dos versiones de \emph{funnel plots} creados con la función \texttt{viz\_sunset} (Fig. \ref{fig:funnel-plot3}). En ambos casos, agregué el efecto \emph{real} encontrado con el meta-análisis (\texttt{contours\ =\ TRUE}), y transformé los tamaños de efecto de regreso a \(r\) de Pearson (\texttt{x\_trans\_function\ =\ tanh}).

\begin{Shaded}
\begin{Highlighting}[]
\CommentTok{\# A. Escala de poder discreta}
\FunctionTok{viz\_sunset}\NormalTok{(res,}
           \AttributeTok{contours =} \ConstantTok{TRUE}\NormalTok{,}
           \AttributeTok{x\_trans\_function =}\NormalTok{ tanh,}
           \AttributeTok{ylab =} \StringTok{"Error estándar"}\NormalTok{,}
           \AttributeTok{xlab =} \StringTok{"Coeficiente de correlación"}\NormalTok{)}

\CommentTok{\# B. Escala de poder contínua}
\FunctionTok{viz\_sunset}\NormalTok{(res, }
           \AttributeTok{contours =} \ConstantTok{TRUE}\NormalTok{,}
           \AttributeTok{x\_trans\_function =}\NormalTok{ tanh, }
           \AttributeTok{power\_contours =} \StringTok{"continuous"}\NormalTok{,}
           \AttributeTok{ylab =} \StringTok{"Error estándar"}\NormalTok{,}
           \AttributeTok{xlab =} \StringTok{"Coeficiente de correlación"}\NormalTok{) }\SpecialCharTok{+}
  \FunctionTok{labs}\NormalTok{(}\AttributeTok{fill =} \StringTok{"Poder"}\NormalTok{)}
\end{Highlighting}
\end{Shaded}

\begin{figure}
\centering
\includegraphics{Meta-analysis_files/figure-latex/funnel-plot3-1.pdf}
\caption{\label{fig:funnel-plot3}Dos versiones de funnel plot creados con \href{https://cran.r-project.org/web/packages/metaviz/vignettes/metaviz.html}{metaviz}, usando la función viz-sunset, que estima el poder de cada estudio para detectar un efecto de interés. \textbf{A.} Poder representado por bandas dicretas de color. \textbf{B.} Poder representado de manera contínua en una escala de color. En ambos casos, y tal como en la Fig. \ref{fig:funnel-plot2}, el efecto real está representado como una línea vertical, y el triángulo a partir de su inicio representa el área donde se ubican los estudios que no se diferencian significativamente del resultado del meta-análisis.}
\end{figure}

\hypertarget{reg-egger}{%
\subsubsection{Regresión de Egger}\label{reg-egger}}

Para hacer una prueba formal de sesgo de estudios pequeños (\protect\hyperlink{ref-schwarzerSmallStudyEffectsMetaAnalysis2015}{Schwarzer et al., 2015b}; \protect\hyperlink{ref-sternePublicationRelatedBias2000}{Sterne et al., 2000}), podemos hacer una prueba o regresión de Egger (\protect\hyperlink{ref-eggerBiasMetaanalysisDetected1997}{Egger et al., 1997}). En \href{https://www.metafor-project.org/doku.php}{\texttt{metafor}}, esto se hace con la función \texttt{regtest}, de nuevo usando como argumento el objeto al que asignamos el resultado de nuestro meta-análisis (\texttt{res}).

\begin{Shaded}
\begin{Highlighting}[]
\FunctionTok{regtest}\NormalTok{(res)}
\end{Highlighting}
\end{Shaded}

Como se puede ver, la prueba de Egger no muestra un resultado significativo (\texttt{z\ =\ 1.0216,\ p\ =\ 0.3070}).

\begin{ROut}{Consola de R: Output~\thetcbcounter}
                \begin{footnotesize}
                \begin{verbatim} 
Regression Test for Funnel Plot Asymmetry

Model:     mixed-effects meta-regression model
Predictor: standard error

Test for Funnel Plot Asymmetry: z = 1.0216, p = 0.3070
Limit Estimate (as sei -> 0):   b = 0.0790 (CI: -0.0686, 0.2266)
 \end{verbatim}
                \end{footnotesize}
                \end{ROut}

Con base en esto, y la inspección visual subjetiva del \emph{funnel plot}, muchos investigadores concluyen que no hay sesgo de publicación. Sin embargo, como mencioné antes, estas pruebas no se centran en el sesgo de publicación sino en el sesgo de estudios pequeños. En otras palabras, con base en esto, lo único que podemos concluir correctamente, es que no hay sesgo de estudios pequeños (más adelante, en la sección \ref{sesgo-pub}, explicaré cómo evaluar si hay sesgo de publicación).

\hypertarget{muxe9todo-trim-and-fill-recorte-y-relleno}{%
\subsubsection{\texorpdfstring{Método \emph{trim and fill} (recorte y relleno)}{Método trim and fill (recorte y relleno)}}\label{muxe9todo-trim-and-fill-recorte-y-relleno}}

\begin{Shaded}
\begin{Highlighting}[]
\FunctionTok{trimfill}\NormalTok{(res)}
\end{Highlighting}
\end{Shaded}

\begin{ROut}{Consola de R: Output~\thetcbcounter}
                \begin{footnotesize}
                \begin{verbatim} 
Estimated number of missing studies on the left side: 2 (SE = 2.7118)

Random-Effects Model (k = 18; tau^2 estimator: REML)

tau^2 (estimated amount of total heterogeneity): 0.0112 (SE = 0.0066)
tau (square root of estimated tau^2 value):      0.1061
I^2 (total heterogeneity / total variability):   67.50%
H^2 (total variability / sampling variability):  3.08

Test for Heterogeneity:
Q(df = 17) = 46.3990, p-val = 0.0002

Model Results:

estimate      se    zval    pval   ci.lb   ci.ub 
  0.1288  0.0333  3.8628  0.0001  0.0635  0.1942  *** 

---
Signif. codes:  0 '***' 0.001 '**' 0.01 '*' 0.05 '.' 0.1 ' ' 1
 \end{verbatim}
                \end{footnotesize}
                \end{ROut}

\begin{Shaded}
\begin{Highlighting}[]
\FunctionTok{funnel}\NormalTok{(}\FunctionTok{trimfill}\NormalTok{(res), }
       \AttributeTok{xlab =} \StringTok{"Coeficiente de correlación transformado en z de Fisher"}\NormalTok{,}
       \AttributeTok{ylab =} \StringTok{"Error estándar"}\NormalTok{)}
\end{Highlighting}
\end{Shaded}

\begin{figure}
\centering
\includegraphics{Meta-analysis_files/figure-latex/unnamed-chunk-20-1.pdf}
\caption{\label{fig:unnamed-chunk-20}En negro los estudios meta-analizados; en blanco, los estudios \emph{rellenados}.}
\end{figure}

\begin{Shaded}
\begin{Highlighting}[]
\FunctionTok{viz\_funnel}\NormalTok{(res, }
           \AttributeTok{contours\_col =} \StringTok{"Oranges"}\NormalTok{,}
           \AttributeTok{trim\_and\_fill =} \ConstantTok{TRUE}\NormalTok{, }
           \AttributeTok{trim\_and\_fill\_side =} \StringTok{"left"}\NormalTok{, }\CommentTok{\#IMPORTANTE}
           \AttributeTok{egger =} \ConstantTok{TRUE}\NormalTok{,}
           \AttributeTok{x\_trans\_function =}\NormalTok{ tanh,}
           \AttributeTok{ylab =} \StringTok{"Error estándar"}\NormalTok{,}
           \AttributeTok{xlab =} \StringTok{"Coeficiente de correlación"}\NormalTok{) }\SpecialCharTok{+}
  \FunctionTok{geom\_vline}\NormalTok{(}\AttributeTok{xintercept =} \DecValTok{0}\NormalTok{, }\AttributeTok{linetype =} \StringTok{"dotted"}\NormalTok{)}
\end{Highlighting}
\end{Shaded}

\includegraphics{Meta-analysis_files/figure-latex/unnamed-chunk-21-1.pdf}

\hypertarget{meta-anuxe1lisis-de-correlaciuxf3n-con-moderador}{%
\section{Meta-análisis de correlación con moderador}\label{meta-anuxe1lisis-de-correlaciuxf3n-con-moderador}}

\hypertarget{ejemplo-1-moderaciuxf3n-de-la-edad-promedio-de-los-participantes}{%
\subsection{Ejemplo 1: Moderación de la edad promedio de los participantes}\label{ejemplo-1-moderaciuxf3n-de-la-edad-promedio-de-los-participantes}}

Primero, y como ejemplo, vamos a ver si la edad (en nuestros datos, \texttt{meanage}) modera el resultado. Esto es importante, pues hay una enorme variación entre las edades medias de los participantes de los diferentes estudios\footnote{De hecho, mientras que en el estudio de Axelsson et al.~(2009) la edad promedio fue de 22, en el estudio de Jerant et al.~(2011) la edad promedio fue de 78.6.}, lo que podría moderar (afectar) la asociación entre diligencia (\emph{conscientiousness}) y adherencia a la medicación prescrita.

Para esto, de nuevo podemos usar la función \texttt{rma} de paquete \texttt{metafor} y de la misma manera que en la sección \ref{meta-cor}, pero agregando nuestra variable moderadora (\texttt{meanage}) al argumento \texttt{mods}. En este caso voy a asignar a un objeto llamado \texttt{res.modage}, para diferenciarlo del objeto \texttt{res} al que asigné el meta-análisis básico, sin moderadores.

\begin{Shaded}
\begin{Highlighting}[]
\NormalTok{res.modage }\OtherTok{\textless{}{-}} \FunctionTok{rma}\NormalTok{(}\AttributeTok{yi =}\NormalTok{ yi, }\AttributeTok{vi =}\NormalTok{ vi, }\AttributeTok{mods =} \SpecialCharTok{\textasciitilde{}}\NormalTok{meanage, }\AttributeTok{data =}\NormalTok{ dat)}
\end{Highlighting}
\end{Shaded}

Los resultados, son los siguientes:

\begin{Shaded}
\begin{Highlighting}[]
\NormalTok{res.modage}
\end{Highlighting}
\end{Shaded}

\begin{ROut}{Consola de R: Output~\thetcbcounter}
                \begin{footnotesize}
                \begin{verbatim} 
Mixed-Effects Model (k = 16; tau^2 estimator: REML)

tau^2 (estimated amount of residual heterogeneity):     0.0072 (SE = 0.0054)
tau (square root of estimated tau^2 value):             0.0846
I^2 (residual heterogeneity / unaccounted variability): 56.50%
H^2 (unaccounted variability / sampling variability):   2.30
R^2 (amount of heterogeneity accounted for):            11.76%

Test for Residual Heterogeneity:
QE(df = 14) = 30.9050, p-val = 0.0057

Test of Moderators (coefficient 2):
QM(df = 1) = 1.4286, p-val = 0.2320

Model Results:

         estimate      se     zval    pval    ci.lb   ci.ub 
intrcpt    0.2741  0.1090   2.5147  0.0119   0.0605  0.4877  * 
meanage   -0.0024  0.0020  -1.1952  0.2320  -0.0063  0.0015    

---
Signif. codes:  0 '***' 0.001 '**' 0.01 '*' 0.05 '.' 0.1 ' ' 1
 \end{verbatim}
                \end{footnotesize}
                \end{ROut}

Los resultados, que tienen la misma organización que los del análisis sin moderadores (sección \ref{meta-cor}) resultado nos muestra que, a pesar de la gran diferencia de edad entre estudios, la edad no tiene un efecto significativo, como se puede ver en la columna \texttt{pval} para el efecto de \texttt{meanage} (0.232).

\hypertarget{plot-mod}{%
\subsubsection{\texorpdfstring{\emph{Forest plot} y \emph{funnel plot}}{Forest plot y funnel plot}}\label{plot-mod}}

Por supuesto, de estos resultados también puedo crear \emph{forest plots} y \emph{funnel plots}, siguiendo los ejemplos y código de la sección \ref{meta-cor}. Para el \emph{forest plot}, hago a continuación un ejemplo anotado y mejorado (Fig. \ref{fig:for-plot-mod1}, con un código similar al usado como ejemplo en la Fig. \ref{fig:for-plot2}). Sin embargo, es importante tener en cuenta que esta opción no creará un resumen del meta-análisis, ya que no tenemos un solo efecto \emph{real} como producto del meta-análisis.

\begin{Shaded}
\begin{Highlighting}[]
\CommentTok{\# forest plot con anotaciones adicionales}
\FunctionTok{forest}\NormalTok{(res.modage,  }\AttributeTok{cex =} \FloatTok{0.75}\NormalTok{, }\AttributeTok{xlim =} \FunctionTok{c}\NormalTok{(}\SpecialCharTok{{-}}\FloatTok{1.6}\NormalTok{, }\FloatTok{1.6}\NormalTok{),}
       \AttributeTok{slab =} \FunctionTok{paste}\NormalTok{(dat}\SpecialCharTok{$}\NormalTok{authors, dat}\SpecialCharTok{$}\NormalTok{year, }\AttributeTok{sep =} \StringTok{", "}\NormalTok{),}
       \AttributeTok{showweights =} \ConstantTok{TRUE}\NormalTok{,}
       \AttributeTok{xlab =} \StringTok{"Coeficiente de correlación transformado en z de Fisher"}\NormalTok{,}
       \AttributeTok{digits =} \FunctionTok{c}\NormalTok{(}\DecValTok{2}\NormalTok{,3L))}
\CommentTok{\# agregar encabezados a las columnas (valores de X y Y deben ser ajustados)}
\FunctionTok{par}\NormalTok{(}\AttributeTok{cex =} \FloatTok{0.8}\NormalTok{, }\AttributeTok{font=}\DecValTok{2}\NormalTok{)}
\FunctionTok{text}\NormalTok{(}\AttributeTok{x =} \SpecialCharTok{{-}}\FloatTok{1.6}\NormalTok{, }\AttributeTok{y =} \DecValTok{18}\NormalTok{, }\AttributeTok{labels =} \StringTok{"Autor(es), Año"}\NormalTok{, }\AttributeTok{pos =} \DecValTok{4}\NormalTok{)}
\FunctionTok{text}\NormalTok{(}\AttributeTok{x =} \DecValTok{0}\NormalTok{, }\AttributeTok{y =} \DecValTok{18}\NormalTok{, }\AttributeTok{labels =} \StringTok{"Efecto e IC"}\NormalTok{, }\AttributeTok{pos =} \DecValTok{4}\NormalTok{)}
\FunctionTok{text}\NormalTok{(}\AttributeTok{x =} \DecValTok{1}\NormalTok{, }\AttributeTok{y =} \DecValTok{18}\NormalTok{, }\AttributeTok{labels =} \StringTok{"Peso"}\NormalTok{, }\AttributeTok{pos =} \DecValTok{2}\NormalTok{)}
\FunctionTok{text}\NormalTok{(}\AttributeTok{x =} \FloatTok{1.6}\NormalTok{, }\AttributeTok{y =} \DecValTok{18}\NormalTok{, }\AttributeTok{labels =} \StringTok{"Corr. [95\% IC]"}\NormalTok{, }\AttributeTok{pos =} \DecValTok{2}\NormalTok{)}
\end{Highlighting}
\end{Shaded}

\begin{figure}
\centering
\includegraphics{Meta-analysis_files/figure-latex/for-plot-mod1-1.pdf}
\caption{\label{fig:for-plot-mod1}Forest plot básico de \href{https://www.metafor-project.org/doku.php}{metafor}, para un meta-análisis incluyendo la edad promedio de los participantes como moderador. En la ilustración gráfica, además de los efectos originales, se puede ver el efecto de cada estudio estimado cuando se incluye el moderador como polígonos (diamantes) de color gris. Sin embargo, ya no obtenemos una fila al final representando el efecto promediado del meta-análisis, ya que no tenemos un solo efecto.}
\end{figure}

Es importante tener en cuenta que la función \href{https://cran.r-project.org/web/packages/metaviz/vignettes/metaviz.html\#creating-forest-plots-with-function-viz_forest}{\texttt{viz\_forest}} del paquete \href{https://cran.r-project.org/web/packages/metaviz/vignettes/metaviz.html}{\texttt{metaviz}} tendrá problemas para crear un \emph{forest plot} de un meta-análisis con moderadores).

De manera similar, podemos obtener un \emph{funnel plot} de nuestro meta-análisis con moderador, pero éste nos mostrará, en vez de los coeficientes de correlación (transformados a \(z\) de Fisher), los valores residuales de cada estudio (es decir, qué tanto se alejan del resultado de nuestro meta-análisis; Fig. \ref{fig:funnel-plot1}):

\begin{Shaded}
\begin{Highlighting}[]
\FunctionTok{funnel}\NormalTok{(res.modage,}
       \AttributeTok{xlab =} \StringTok{"Valor residual"}\NormalTok{,}
       \AttributeTok{ylab =} \StringTok{"Error estándar"}\NormalTok{)}
\end{Highlighting}
\end{Shaded}

\begin{figure}
\centering
\includegraphics{Meta-analysis_files/figure-latex/funnel-plot-mod1-1.pdf}
\caption{\label{fig:funnel-plot-mod1}Funnel plot básico de \href{https://www.metafor-project.org/doku.php}{metafor}, para un meta-análisis incluyendo la edad promedio de los participantes como moderador, y con títulos de los ejes en español. La línea punteada vertical representa el efecto meta-analizado que hemos encontrado, así que podemos ver los estudios que encontraron un efecto mayor (derecha de la línea punteada) o menor (izquierda) de éste.}
\end{figure}

\hypertarget{meta-analytic-scatter-plot-gruxe1fico-de-dispersiuxf3n-meta-analuxedtico}{%
\subsubsection{\texorpdfstring{\emph{Meta-Analytic Scatter Plot} (Gráfico de dispersión meta-analítico)}{Meta-Analytic Scatter Plot (Gráfico de dispersión meta-analítico)}}\label{meta-analytic-scatter-plot-gruxe1fico-de-dispersiuxf3n-meta-analuxedtico}}

Cuando tenemos un modelo con moderador, también podemos ver la asociación entre la variable moderadora, y el efecto de cada estudio meta-analizado, a modo de regresión. La función \texttt{regplot}hace precisamente esto (Fig. @(fig:reg-plot)).

\begin{Shaded}
\begin{Highlighting}[]
\FunctionTok{regplot}\NormalTok{(res.modage)}
\end{Highlighting}
\end{Shaded}

\begin{figure}
\centering
\includegraphics{Meta-analysis_files/figure-latex/reg-plot-1.pdf}
\caption{\label{fig:reg-plot}Gráfico de dispersión meta-analítico (\emph{Meta-Analytic Scatter Plot}). El tamaño de los puntos es proporcional al peso que recibieron los estudios en el meta-análisis (puntos más grandes para los estudios con más peso). La línea negra representa el efecto previsto en función del predictor (en este caso \texttt{age}, edad) con intervalo de confianza del 95\%.}
\end{figure}

\hypertarget{ejemplo-2-moderaciuxf3n-de-la-calidad-de-los-estudios-meta-analizados}{%
\subsection{Ejemplo 2: Moderación de la calidad de los estudios meta-analizados}\label{ejemplo-2-moderaciuxf3n-de-la-calidad-de-los-estudios-meta-analizados}}

La base de datos con tiene una medida de calidad metodológica de los estudios (variable \texttt{quality}). Dicha calidad, también podría moderar la asociación entre diligencia (\emph{conscientiousness}) y adherencia a la medicación prescrita. Siguiendo los mismos pasos, puedo hacer éste análisis, pero voy a asignar este meta-análisis a un objeto llamado \texttt{res.modq} para diferenciarlo de los demás.

\begin{Shaded}
\begin{Highlighting}[]
\NormalTok{res.modq }\OtherTok{\textless{}{-}} \FunctionTok{rma}\NormalTok{(}\AttributeTok{yi =}\NormalTok{ yi, }\AttributeTok{vi =}\NormalTok{ vi, }\AttributeTok{mods =} \SpecialCharTok{\textasciitilde{}}\NormalTok{quality, }\AttributeTok{data =}\NormalTok{ dat)}
\NormalTok{res.modq}
\end{Highlighting}
\end{Shaded}

\begin{ROut}{Consola de R: Output~\thetcbcounter}
                \begin{footnotesize}
                \begin{verbatim} 
Mixed-Effects Model (k = 16; tau^2 estimator: REML)

tau^2 (estimated amount of residual heterogeneity):     0.0078 (SE = 0.0057)
tau (square root of estimated tau^2 value):             0.0884
I^2 (residual heterogeneity / unaccounted variability): 57.79%
H^2 (unaccounted variability / sampling variability):   2.37
R^2 (amount of heterogeneity accounted for):            3.73%

Test for Residual Heterogeneity:
QE(df = 14) = 30.4205, p-val = 0.0067

Test of Moderators (coefficient 2):
QM(df = 1) = 0.6393, p-val = 0.4240

Model Results:

         estimate      se     zval    pval    ci.lb   ci.ub 
intrcpt    0.2082  0.0796   2.6149  0.0089   0.0521  0.3643  ** 
quality   -0.0312  0.0391  -0.7995  0.4240  -0.1078  0.0453     

---
Signif. codes:  0 '***' 0.001 '**' 0.01 '*' 0.05 '.' 0.1 ' ' 1
 \end{verbatim}
                \end{footnotesize}
                \end{ROut}

De nuevo, encontramos que éste moderador (\texttt{quality}), al igual que la edad promedio (\texttt{meanage}), no tiene un efecto significativo, como se puede ver en la columna \texttt{pval} para el efecto de \texttt{quality} (0.424).

Por supuesto, \emph{forest plots} y \emph{funnel plots} pueden ser creados, tal y como describí en la sección \ref{plot-mod}.

\hypertarget{ejemplo-3-moderaciuxf3n-de-las-controles-usados-en-cada-estudio-meta-analizado}{%
\subsection{Ejemplo 3: Moderación de las controles usados en cada estudio meta-analizado}\label{ejemplo-3-moderaciuxf3n-de-las-controles-usados-en-cada-estudio-meta-analizado}}

Como último ejemplo, voy a mirar si el hecho de que los estudios tengan variables que fueron controladas, modera la asociación entre diligencia (\emph{conscientiousness}) y adherencia a la medicación prescrita. Siguiendo los mismos pasos, voy hacer éste análisis, pero voy a asignar este meta-análisis a un objeto llamado \texttt{res.mes}. Son embargo, dado que la variable que contiene esta información (\texttt{controls}) es un factor, pero no está definido como tal, debo hacerlo en la usando la función \texttt{factor} al ingresar el argumento \texttt{mods} (i.e.~\texttt{mods\ =\ \textasciitilde{}factor(controls)}).

\begin{Shaded}
\begin{Highlighting}[]
\NormalTok{res.mes }\OtherTok{\textless{}{-}} \FunctionTok{rma}\NormalTok{(}\AttributeTok{yi =}\NormalTok{ yi, }\AttributeTok{vi =}\NormalTok{ vi, }\AttributeTok{mods =} \SpecialCharTok{\textasciitilde{}}\FunctionTok{factor}\NormalTok{(controls), }\AttributeTok{data =}\NormalTok{ dat)}
\NormalTok{res.mes}
\end{Highlighting}
\end{Shaded}

\begin{ROut}{Consola de R: Output~\thetcbcounter}
                \begin{footnotesize}
                \begin{verbatim} 
Mixed-Effects Model (k = 16; tau^2 estimator: REML)

tau^2 (estimated amount of residual heterogeneity):     0.0000 (SE = 0.0015)
tau (square root of estimated tau^2 value):             0.0002
I^2 (residual heterogeneity / unaccounted variability): 0.00%
H^2 (unaccounted variability / sampling variability):   1.00
R^2 (amount of heterogeneity accounted for):            100.00%

Test for Residual Heterogeneity:
QE(df = 14) = 18.0370, p-val = 0.2051

Test of Moderators (coefficient 2):
QM(df = 1) = 20.1221, p-val < .0001

Model Results:

                      estimate      se    zval    pval    ci.lb   ci.ub 
intrcpt                 0.0167  0.0296  0.5635  0.5731  -0.0413  0.0746      
factor(controls)none    0.1621  0.0361  4.4858  <.0001   0.0913  0.2329  *** 

---
Signif. codes:  0 '***' 0.001 '**' 0.01 '*' 0.05 '.' 0.1 ' ' 1
 \end{verbatim}
                \end{footnotesize}
                \end{ROut}

En éste caso, a diferencia de los ejemplos de moderación anteriores, la variable moderadora (\texttt{controls}) sí tiene un efecto significativo, como se puede ver en la columna \texttt{pval} para el efecto de \texttt{factor(controls)none} (\textless0.001), y en los asteriscos que aparecen al final de esa fila (\texttt{***}).

Por supuesto, \emph{forest plots} y \emph{funnel plots} pueden ser creados, tal y como describí en la sección \ref{plot-mod}.

\hypertarget{sesgo-pub}{%
\section{\texorpdfstring{Sesgo de publicación (\emph{publication bias})}{Sesgo de publicación (publication bias)}}\label{sesgo-pub}}

Para determinar el sesgo de publicación, se puede usar la función \texttt{weightfunct} del paquete \texttt{\{weightr\}}, que nos permite ``estimar tanto el modelo de función de peso para el sesgo de publicación que se publicó originalmente en Vevea y Hedges (\protect\hyperlink{ref-veveaGeneralLinearModel1995}{1995}) como la versión modificada presentada en Vevea y Woods (\protect\hyperlink{ref-veveaPublicationBiasResearch2005}{2005})'', como se describe en la \href{https://www.rdocumentation.org/packages/weightr/versions/2.0.2/topics/weightfunct}{documentación} de la función \texttt{weightfunct}.

\begin{Shaded}
\begin{Highlighting}[]
\FunctionTok{library}\NormalTok{(weightr)}
\end{Highlighting}
\end{Shaded}

En este caso, usaré esta función, asignando el resultado a un objeto que llamaré \texttt{wf}.

\begin{Shaded}
\begin{Highlighting}[]
\NormalTok{wf }\OtherTok{\textless{}{-}} \FunctionTok{weightfunct}\NormalTok{(}\AttributeTok{effect =}\NormalTok{ dat}\SpecialCharTok{$}\NormalTok{yi, }\AttributeTok{v =}\NormalTok{ dat}\SpecialCharTok{$}\NormalTok{vi, }\AttributeTok{table =} \ConstantTok{TRUE}\NormalTok{)}
\NormalTok{wf}
\end{Highlighting}
\end{Shaded}

\begin{ROut}{Consola de R: Output~\thetcbcounter}
                \begin{footnotesize}
                \begin{verbatim} 
Unadjusted Model (k = 16):

tau^2 (estimated amount of total heterogeneity): 0.0070 (SE = 0.0051)
tau (square root of estimated tau^2 value):  0.0834

Test for Heterogeneity:
Q(df = 15) = 38.1595, p-val = 0.001436053

Model Results:

          estimate std.error z-stat      p-val   ci.lb  ci.ub
Intercept   0.1486   0.03073  4.835 1.3288e-06 0.08837 0.2088

Adjusted Model (k = 16):

tau^2 (estimated amount of total heterogeneity): 0.0056 (SE = 0.0045)
tau (square root of estimated tau^2 value):  0.0750

Test for Heterogeneity:
Q(df = 15) = 38.1595, p-val = 0.001436053

Model Results:

              estimate std.error z-stat    p-val    ci.lb  ci.ub
Intercept      0.09153   0.04464  2.050 0.040341  0.00403 0.1790
0.025 < p < 1  0.24121   0.20122  1.199 0.230626 -0.15317 0.6356

Likelihood Ratio Test:
X^2(df = 1) = 2.98493, p-val = 0.084043

Number of Effect Sizes per Interval:

                     Frequency
p-values <0.025              9
0.025 < p-values < 1         7
 \end{verbatim}
                \end{footnotesize}
                \end{ROut}

El modelo tradicional nos da un estimado muy similar al des estudio original (0.1486), dado que usa un método ligeramente diferente.

Nos da valores de heterogeneidad \(\tau^2\), \(\tau\) y \(Q\).

Pero lo más importante, es que nos da los resultado del meta-análisis, ajustando los pesos dados a cada efecto, de cada estudio meta-analizado.

Lo que esta función hace es lo que se conoce como \emph{selection models} (modelos de selección). Básicamente, da más peso a ciertos tamaños de efecto. La realidad de la literatura científica es que es más probable que algunos estudios sean publicados, dependiendo de sus valores \(p\) (ver SESGO DE PUBLICACIÓN); estudios con valores \(p\) mayores a 0.05 (o 0.10), tienen menos probabilidad de ser publicados que estudios con \(p < 0.05\).

La función \texttt{weightfunct} incrementa el peso de estudios que tienen menos probabilidad de ser publicados, y reduce el peso de estudios con mayor probabilidad de ser publicados. Por esto, al usar ésta técnica, estás asumiendo que de hecho, en el efecto que tratas de encontrar en tu meta-análisis, de hecho hay un sesgo de publicación, lo que a menudo es una suposición bastante justa.

Al usar ésta técnica, tenemos un resultado bastante distinto. Mientras que el meta-análisis original nos daba como resultado un efecto de \textasciitilde0.15, esta técnica nos estima un efecto de \textasciitilde0.09. Básicamente, a \emph{encogido} nuestro tamaño de efecto.

Al final el \emph{Likelihood ratio test} (algo así como ``Prueba de cociente de probabilidades''), que evalúa la bondad del ajuste de dos modelos estadísticos que compiten entre sí basándose en la relación de su verosimilitud. En este caso, comparando el modelo original, con este modelo con pesos ajustados.

Este resultado nos da una tendencia no descartable (p-val = 0.084043, lo que es \textless{} 0.10; significativa si asumimos un análisis de una cola), que nos da evidencia de que en efecto hay un sesgo de publicación, a pesar de que el \emph{funnel plot} (Figs. \ref{fig:funnel-plot1}, \ref{fig:funnel-plot1a} \ref{fig:funnel-plot2} y \ref{fig:funnel-plot3}) y la regresión de Egger (sección \ref{reg-egger}), sugerían lo contrario.

\hypertarget{poder-estaduxedstico-del-meta-anuxe1lisis}{%
\section{Poder estadístico del meta-análisis}\label{poder-estaduxedstico-del-meta-anuxe1lisis}}

En esta sección explicaré cómo hacer un análisis de poder de un meta-análisis; la idea de ésto es saber si nuestro meta-análisis tiene un poder suficiente para detectar el efecto meta-analizado (en nuestro caso 0.15 para el meta-análisis original ``\texttt{res}'', o 0.09 el meta-análisis con pesos ajustados ``\texttt{wf}''). Para este ejemplo, asumiré que el efecto \emph{real} es el encontrado en nuestro análisis original (0.15), pues este efecto es más mayor. Si no tuviésemos el poder suficiente para detectar confiablemente ese efecto, menos lo tendríamos para un efecto menor, como el detectado en nuestro meta-análisis con pesos ajustados.

Para hacer esto, usaré el paquete \texttt{metameta} (\protect\hyperlink{ref-quintanaMetameta2022}{Quintana, 2022}), que permite calcular y visualizar el poder estadístico de un meta-análisis para detectar un rango de posibles efectos \emph{reales}.

\hypertarget{instalaciuxf3n-de-metameta}{%
\subsection{\texorpdfstring{Instalación de \texttt{metameta}}{Instalación de metameta}}\label{instalaciuxf3n-de-metameta}}

El paquete \texttt{metameta} se debe instalar desde GitHub\footnote{\href{https://github.com/}{GitHub} es un repositorio abierto para para proyectos de código abierto en el que, entre otras cosas, suelen estar alojados todos los paquetes de R incluso en versiones de desarrollador. Por supuesto, a diferencia de \href{https://cran.r-project.org/}{CRAN}, GitHub no es ni mucho menos específico para paquetes de R.} ya que, al día de hoy, no está aún disponible en CRAN.

Para esto, debemos tener instalado el paquete \texttt{devtools}, y usar la función \texttt{install\_github} que nos permite instalar paquetes directamente desde GitHub.

\begin{Shaded}
\begin{Highlighting}[]
\CommentTok{\#se debe tener instalado el paquete devtools}
\FunctionTok{library}\NormalTok{(devtools)}
\FunctionTok{install\_github}\NormalTok{(}\StringTok{"dsquintana/metameta"}\NormalTok{)}
\end{Highlighting}
\end{Shaded}

\hypertarget{anuxe1lisis-de-poder}{%
\subsection{Análisis de poder}\label{anuxe1lisis-de-poder}}

Una vez instalado, podemos cargar el paquete.

\begin{Shaded}
\begin{Highlighting}[]
\FunctionTok{library}\NormalTok{(metameta)}
\end{Highlighting}
\end{Shaded}

Como datos, necesitamos no solamente los tamaños de datos a meta-analizar (\(r\) de Pearson transformado a \emph{z} de Fisher), sino además los intervalos de confianza, tal como fueron reportados en varios de nuestros \emph{Forest plots}.

Debemos asumir el efecto real, pero esto nunca podemos saberlo. En este caso, asumimos un efecto de \(r = 0.15\), tal como en nuestro meta-análisis original. Sin embargo, el efecto real no es algo que podamos saber (es, de hecho, lo que queremos acercarnos a conocer a través del meta-análisis), así que la función \texttt{mapower\_ul} del paquete \texttt{metameta} calcula el poder de cada meta-análisis para un rango de posibles efectos reales.

\begin{Shaded}
\begin{Highlighting}[]
\NormalTok{dat.power }\OtherTok{\textless{}{-}} \FunctionTok{summary}\NormalTok{(dat) }\SpecialCharTok{\%\textgreater{}\%}
  \FunctionTok{select}\NormalTok{(yi, ci.lb, ci.ub) }\SpecialCharTok{\%\textgreater{}\%}
  \FunctionTok{rename}\NormalTok{(}\AttributeTok{lower =}\NormalTok{ ci.lb, }\AttributeTok{upper =}\NormalTok{ ci.ub)}
\NormalTok{power }\OtherTok{\textless{}{-}} \FunctionTok{mapower\_ul}\NormalTok{(}\AttributeTok{dat =}\NormalTok{ dat.power, }\AttributeTok{observed\_es =} \FloatTok{0.15}\NormalTok{, }\AttributeTok{name =} \StringTok{"Molloy et al. 2014"}\NormalTok{)}

\NormalTok{power\_dat }\OtherTok{\textless{}{-}}\NormalTok{ power}\SpecialCharTok{$}\NormalTok{dat}
\NormalTok{power\_dat}
\end{Highlighting}
\end{Shaded}

\begin{ROut}{Consola de R: Output~\thetcbcounter}
                \begin{footnotesize}
                \begin{verbatim}             yi        lower      upper        sei power_es_observed power_es01
1   0.18922664 -0.001141888 0.37959517 0.09712680         0.3390835  0.1774836
2   0.16343992  0.091680539 0.23519930 0.03661193         0.9837035  0.7797623
3   0.35409253  0.082294427 0.62589063 0.13867250         0.1910727  0.1113715
4   0.33164711  0.139456828 0.52383739 0.09805627         0.3337559  0.1750148
5   0.27686382  0.040911886 0.51281576 0.12038364         0.2383028  0.1320187
6   0.00000000 -0.248915675 0.24891567 0.12699779         0.2188783  0.1234921
7   0.17682002  0.026937828 0.32670221 0.07647051         0.5006730  0.2576556
8   0.05004173 -0.059013584 0.15909704 0.05564047         0.7691106  0.4354589
9   0.26610841  0.001826917 0.53038990 0.13483749         0.1994161  0.1149996
10  0.01000033 -0.060723775 0.08072444 0.03608373         0.9859910  0.7914240
11 -0.09024419 -0.359465946 0.17897757 0.13735804         0.1938536  0.1125800
12  0.38842310  0.179490237 0.59735596 0.10659840         0.2905754  0.1553001
13  0.00000000 -0.184377902 0.18437790 0.09407036         0.3575910  0.1861296
14  0.15114044  0.066324499 0.23595637 0.04327344         0.9340134  0.6371865
15  0.24477411  0.087346017 0.40220221 0.08032046         0.4632371  0.2379950
16  0.04002135 -0.208894321 0.28893703 0.12699779         0.2188783  0.1234921
   power_es02 power_es03 power_es04 power_es05 power_es06 power_es07 power_es08
1   0.5395394  0.8705053  0.9845502  0.9992836  0.9999877  0.9999999  1.0000000
2   0.9997697  1.0000000  1.0000000  1.0000000  1.0000000  1.0000000  1.0000000
3   0.3026621  0.5806100  0.8223954  0.9500825  0.9910281  0.9989921  0.9999302
4   0.5317864  0.8642261  0.9829686  0.9991528  0.9999840  0.9999999  1.0000000
5   0.3827659  0.7026651  0.9135189  0.9858616  0.9987531  0.9999421  0.9999986
6   0.3502733  0.6562694  0.8829174  0.9759855  0.9971497  0.9998088  0.9999929
7   0.7439048  0.9751837  0.9994638  0.9999977  1.0000000  1.0000000  1.0000000
8   0.9489275  0.9997002  0.9999999  1.0000000  1.0000000  1.0000000  1.0000000
9   0.3170763  0.6044851  0.8429297  0.9597856  0.9936099  0.9993842  0.9999645
10  0.9998300  1.0000000  1.0000000  1.0000000  1.0000000  1.0000000  1.0000000
11  0.3074782  0.5886809  0.8294859  0.9535367  0.9919839  0.9991443  0.9999443
12  0.4666848  0.8035418  0.9634686  0.9968384  0.9998781  0.9999980  1.0000000
13  0.5659845  0.8904902  0.9890522  0.9996035  0.9999950  1.0000000  1.0000000
14  0.9961139  0.9999997  1.0000000  1.0000000  1.0000000  1.0000000  1.0000000
15  0.7019697  0.9620572  0.9987365  0.9999900  1.0000000  1.0000000  1.0000000
16  0.3502733  0.6562694  0.8829174  0.9759855  0.9971497  0.9998088  0.9999929
   power_es09 power_es1
1   1.0000000 1.0000000
2   1.0000000 1.0000000
3   0.9999971 0.9999999
4   1.0000000 1.0000000
5   1.0000000 1.0000000
6   0.9999999 1.0000000
7   1.0000000 1.0000000
8   1.0000000 1.0000000
9   0.9999988 1.0000000
10  1.0000000 1.0000000
11  0.9999978 0.9999999
12  1.0000000 1.0000000
13  1.0000000 1.0000000
14  1.0000000 1.0000000
15  1.0000000 1.0000000
16  0.9999999 1.0000000
 \end{verbatim}
                \end{footnotesize}
                \end{ROut}

\hypertarget{visualizaciuxf3n-del-anuxe1lisis-de-poder-firepower-plot}{%
\subsection{\texorpdfstring{Visualización del análisis de poder (\emph{Firepower plot})}{Visualización del análisis de poder (Firepower plot)}}\label{visualizaciuxf3n-del-anuxe1lisis-de-poder-firepower-plot}}

Según esto, nuestro meta-análisis solo tiene un poder estadístico suficiente para detectar de manera confiable efectos mayores a 0.4, lo que está muy por encima de nuestras estimaciones de un efecto real (0.15 en nuestro meta-análisis original, 0.09 en nuestro meta-análisis con pesos ajustados)

\begin{Shaded}
\begin{Highlighting}[]
\NormalTok{power\_list }\OtherTok{\textless{}{-}} \FunctionTok{list}\NormalTok{(power}\SpecialCharTok{$}\NormalTok{power\_median\_dat)}
\NormalTok{power.plot }\OtherTok{\textless{}{-}} \FunctionTok{firepower}\NormalTok{(power\_list)}
\end{Highlighting}
\end{Shaded}

Para ver el \emph{fireplot} que creamoe, y ya que lo asigné a un objeto que llamé \texttt{power.plot}, debo correr el objeto para ver el resultado (Fig. \ref{fig:fire-plot1}).

\begin{Shaded}
\begin{Highlighting}[]
\NormalTok{power.plot}
\end{Highlighting}
\end{Shaded}

\begin{figure}
\centering
\includegraphics{Meta-analysis_files/figure-latex/fire-plot1-1.pdf}
\caption{\label{fig:fire-plot1}Fireplot básico de \href{https://www.dsquintana.blog/metameta-r-package-meta-analysis/}{metameta}, para un análisis de poder de nuestro meta-análisis. \emph{Observed} hace referencia al tamaño de efecto observado en nuestro meta-análisis original; en este caso, 0.15.}
\end{figure}

Si queremos cambiar los títulos a español, y ya que el objeto \texttt{power.plot} contiene dos elementos (\texttt{dat} y \texttt{fp\_plot}, que es propiamente la gráfica). Éste último elemento es de clase \texttt{ggplot}, por lo que podemos usar funciones de \texttt{ggplot2} para cambiar, por ejemplo, el título del eje \(X\) a ``Tamaño de efecto'', el título de la leyenda a ``Poder'', y el efecto observado de ``\emph{Observed}'' a ``Observado''\footnote{Para cambiar el título del eje \(X\) usé la función \texttt{xlab}; para el título de la leyenda la función \texttt{guides} (opción \texttt{fill\ =\ guide\_legend}); y para los valores del eje \(X\), la función \texttt{scale\_x\_discrete}\}.} (Fig. \ref{fig:fire-plot2}). Por ejemplo:

\begin{Shaded}
\begin{Highlighting}[]
\NormalTok{power.plot}\SpecialCharTok{$}\NormalTok{fp\_plot }\SpecialCharTok{+}
  \FunctionTok{xlab}\NormalTok{(}\StringTok{"Tamaño de efecto"}\NormalTok{) }\SpecialCharTok{+}
  \FunctionTok{guides}\NormalTok{(}\AttributeTok{fill =} \FunctionTok{guide\_legend}\NormalTok{(}\AttributeTok{title =} \StringTok{"Poder"}\NormalTok{, }
                             \AttributeTok{reverse =} \ConstantTok{TRUE}\NormalTok{)) }\SpecialCharTok{+}
  \FunctionTok{scale\_x\_discrete}\NormalTok{(}\AttributeTok{labels =} \FunctionTok{c}\NormalTok{(}\StringTok{"es\_observed"} \OtherTok{=} \StringTok{"Observado"}\NormalTok{,}
                              \StringTok{"es01"} \OtherTok{=} \FloatTok{0.1}\NormalTok{,    }
                              \StringTok{"es02"} \OtherTok{=} \FloatTok{0.2}\NormalTok{,}
                              \StringTok{"es03"} \OtherTok{=} \FloatTok{0.3}\NormalTok{,}
                              \StringTok{"es04"} \OtherTok{=} \FloatTok{0.4}\NormalTok{,    }
                              \StringTok{"es05"} \OtherTok{=} \FloatTok{0.5}\NormalTok{,    }
                              \StringTok{"es06"} \OtherTok{=} \FloatTok{0.6}\NormalTok{,    }
                              \StringTok{"es07"} \OtherTok{=} \FloatTok{0.7}\NormalTok{,}
                              \StringTok{"es08"} \OtherTok{=} \FloatTok{0.8}\NormalTok{,}
                              \StringTok{"es09"} \OtherTok{=} \FloatTok{0.9}\NormalTok{,}
                              \StringTok{"es1"}  \OtherTok{=}\DecValTok{1}\NormalTok{))}
\end{Highlighting}
\end{Shaded}

\begin{figure}
\centering
\includegraphics{Meta-analysis_files/figure-latex/fire-plot2-1.pdf}
\caption{\label{fig:fire-plot2}Fireplot básico de \href{https://www.dsquintana.blog/metameta-r-package-meta-analysis/}{metameta}, para un análisis de poder de nuestro meta-análisis, con el texto traducido a español y con la leyenda en una escala discreta para facilitar su lectura. \emph{Observado} hace referencia al tamaño de efecto observado en nuestro meta-análisis original (en este caso, 0.15).}
\end{figure}

\newpage

\hypertarget{apuxe9ndices}{%
\section*{APÉNDICES}\label{apuxe9ndices}}
\addcontentsline{toc}{section}{APÉNDICES}

\hypertarget{alternativas-a-metafor}{%
\subsection*{\texorpdfstring{Alternativas a \texttt{metafor}}{Alternativas a metafor}}\label{alternativas-a-metafor}}
\addcontentsline{toc}{subsection}{Alternativas a \texttt{metafor}}

Acá he usado principalmente una ruta para hacer meta-análisis basada en el paquete \texttt{metafor}, acompañado de \texttt{metaviz} para visualizaciones, \texttt{weightr} para ajustar pesos y detectar sesgos de publicación, y \texttt{metameta} para estimar el poder estadístico de un meta-análisis.

Sin embargo, existen rutas alternativas para realizar meta-análisis en R. El libro \emph{Doing meta-analysis with R: a hands-on guide} (\protect\hyperlink{ref-harrer2021doing}{Harrer et al., 2021}) se acompaña del paquete \href{https://dmetar.protectlab.org/index.html}{\texttt{dmetar}} (\protect\hyperlink{ref-Harrer2019dmetar}{Harrer et al., 2019}), que contiene opciones para hacer meta-análisis tanto a partir de \texttt{metafor}, como a partir de \texttt{meta} (\protect\hyperlink{ref-BalduzziMeta2019}{Balduzzi et al., 2019}; \protect\hyperlink{ref-schwarzerMetaAnalysis2015}{Schwarzer et al., 2015a}).

De manera importante, los objetos generados por \texttt{meta} al realizar un meta-análisis permiten hacer otros análisis como \emph{risk of bias} (riesgo de sesgo), inferencia multi-modelo, detección de \emph{outliers} (valores atípicos), o \emph{p-curve} (curva de valores \(p\)), así como opciones para hacer gráficos distintos. Para una guía resumida y concreta (en inglés) de estas opciones, recomiendo ver el sitio web del paquete \href{http://dmetar.protectlab.org/}{\texttt{dmetar}}, y en especial la página \href{https://dmetar.protectlab.org/articles/dmetar.html}{\emph{Get Started}}.

\hypertarget{citas-y-referencias-de-paquetes-de-r}{%
\subsection*{Citas y referencias de paquetes de R}\label{citas-y-referencias-de-paquetes-de-r}}
\addcontentsline{toc}{subsection}{Citas y referencias de paquetes de R}

Por supuesto, los paquetes de R que usemos deben ser citados. Una manera fácil de encontrar la cita que los autores de un paquete quieren que usemos, es la función \texttt{citation} en R. Simplemente debemos usar esta función, agregando como argumento el nombre del paquete que queremos usar entre comillas. Esto nos dará la referencia en un formato estándar, así como como en un formato \texttt{BibTex} que puede ser usado en documentos \LaTeX, o por muchos gestores de referencia (alternativamente nos permite saber los campos como autores, título y demás, si vamos a crear las citas y referencias a mano).

Por ejemplo, en ésta guía usé \texttt{dplyr} (\protect\hyperlink{ref-WickhamDplyr2021}{Wickham et al., 2021}) para manipular los datos, y usando la función \texttt{citation}, obtengo esta información:

\begin{Shaded}
\begin{Highlighting}[]
\FunctionTok{citation}\NormalTok{(}\StringTok{"dplyr"}\NormalTok{)}
\end{Highlighting}
\end{Shaded}

\begin{ROut}{Consola de R: Output~\thetcbcounter}
                \begin{footnotesize}
                \begin{verbatim} 
To cite package 'dplyr' in publications use:

  Hadley Wickham, Romain François, Lionel Henry and Kirill Müller
  (2021). dplyr: A Grammar of Data Manipulation. R package version
  1.0.7. https://CRAN.R-project.org/package=dplyr

A BibTeX entry for LaTeX users is

  @Manual{,
    title = {dplyr: A Grammar of Data Manipulation},
    author = {Hadley Wickham and Romain François and Lionel Henry and Kirill Müller},
    year = {2021},
    note = {R package version 1.0.7},
    url = {https://CRAN.R-project.org/package=dplyr},
  }
 \end{verbatim}
                \end{footnotesize}
                \end{ROut}

\newpage

\hypertarget{referencias}{%
\section*{Referencias}\label{referencias}}
\addcontentsline{toc}{section}{Referencias}

\hypertarget{refs}{}
\begin{CSLReferences}{1}{0}
\leavevmode\vadjust pre{\hypertarget{ref-BalduzziMeta2019}{}}%
Balduzzi, S., Rücker, G., \& Schwarzer, G. (2019). How to perform a meta-analysis with {R}: A practical tutorial. \emph{Evidence-Based Mental Health}, \emph{22}, 153--160. \url{https://doi.org/10.1136/ebmental-2019-300117}

\leavevmode\vadjust pre{\hypertarget{ref-borensteinIdentifyingQuantifyingHeterogeneity2009}{}}%
Borenstein, M., Hedges, L. V., Higgns, J. P. T., \& Rothstein, H. R. (2009). Identifying and {Quantifying Heterogeneity}. In \emph{Introduction to {Meta}-{Analysis}} (pp. 107--125). Wiley. \url{https://doi.org/10.1002/9780470743386.ch16}

\leavevmode\vadjust pre{\hypertarget{ref-eggerBiasMetaanalysisDetected1997}{}}%
Egger, M., Smith, G. D., Schneider, M., \& Minder, C. (1997). {Bias in Meta-Analysis Detected by a Simple, Graphical Test}. \emph{BMJ}, \emph{315}(7109), 629--634. \url{https://doi.org/10.1136/bmj.315.7109.629}

\leavevmode\vadjust pre{\hypertarget{ref-harrer2021doing}{}}%
Harrer, M., Cuijpers, P., A, F. T., \& Ebert, D. D. (2021). \emph{{Doing Meta-Analysis With {R}: A Hands-On Guide}} (1st ed.). Chapman \& Hall/CRC Press. \url{https://bookdown.org/MathiasHarrer/Doing_Meta_Analysis_in_R/}

\leavevmode\vadjust pre{\hypertarget{ref-Harrer2019dmetar}{}}%
Harrer, M., Cuijpers, P., Furukawa, T., \& Ebert, D. D. (2019). \emph{{dmetar: Companion R Package For The Guide 'Doing Meta-Analysis in R'}}. \url{http://dmetar.protectlab.org/}

\leavevmode\vadjust pre{\hypertarget{ref-KossmeierMetaviz}{}}%
Kossmeier, M., Tran, U. S., \& Voracek, M. (2020). \emph{{metaviz: Forest Plots, Funnel Plots, and Visual Funnel Plot Inference for Meta-Analysis}}. \url{https://CRAN.R-project.org/package=metaviz}

\leavevmode\vadjust pre{\hypertarget{ref-leongomezMetaanalysis2021}{}}%
Leongómez, J. D. (2021). \emph{Hacer meta-análisis en jamovi es muy fácil}. {[}Video{]}. YouTube. \url{https://youtu.be/ntBbkOn9D_o}.

\leavevmode\vadjust pre{\hypertarget{ref-leongomezAnalisisPoderEstadistico2020}{}}%
Leongómez, J. D. (2020a). \emph{Análisis de poder estadístico y cálculo de tamaño de muestra en {R}: {Guía} práctica}. {Zenodo}. \url{https://doi.org/10.5281/zenodo.3988776}

\leavevmode\vadjust pre{\hypertarget{ref-leongomezPoderRvid2020}{}}%
Leongómez, J. D. (2020b). \emph{Poder estadístico y tamaño de muestra en {R}}. {[}Serie de Videos{]}. YouTube. \url{https://youtube.com/playlist?list=PLHk7UNt35ccVdyHqnQ6oXVYA6JBNFrE1x}.

\leavevmode\vadjust pre{\hypertarget{ref-molloy2013}{}}%
Molloy, G. J., O'Carroll, R. E., \& Ferguson, E. (2013). Conscientiousness and Medication Adherence: A Meta-analysis. \emph{Annals of Behavioral Medicine}, \emph{47}(1), 92--101. \url{https://doi.org/10.1007/s12160-013-9524-4}

\leavevmode\vadjust pre{\hypertarget{ref-quintanaHowPerformMetaanalysis2021}{}}%
Quintana, D. S. (2021). \emph{{How to Perform a Meta-Analysis in R}}. {[}Video{]}. YouTube. \url{https://youtu.be/lH4VZMTEZSc}.

\leavevmode\vadjust pre{\hypertarget{ref-quintanaMetameta2022}{}}%
Quintana, D. S. (2022). \emph{Metameta: A suite of tools to re-evaluate published meta-analyses}. \url{https://github.com/dsquintana/metameta}

\leavevmode\vadjust pre{\hypertarget{ref-schwarzerMetaAnalysis2015}{}}%
Schwarzer, G., Carpenter, J. R., \& Rücker, G. (2015a). \emph{Meta-{Analysis} with {R}}. {Springer International Publishing}. \url{https://doi.org/10.1007/978-3-319-21416-0}

\leavevmode\vadjust pre{\hypertarget{ref-schwarzerSmallStudyEffectsMetaAnalysis2015}{}}%
Schwarzer, G., Carpenter, J. R., \& Rücker, G. (2015b). Small-{Study Effects} in {Meta}-{Analysis}. In G. Schwarzer, J. R. Carpenter, \& G. Rücker (Eds.), \emph{Meta-{Analysis} with {R}} (pp. 107--141). {Springer International Publishing}. \url{https://doi.org/10.1007/978-3-319-21416-0_5}

\leavevmode\vadjust pre{\hypertarget{ref-sternePublicationRelatedBias2000}{}}%
Sterne, J. A. C., Gavaghan, D., \& Egger, M. (2000). Publication and related bias in meta-analysis. \emph{Journal of Clinical Epidemiology}, \emph{53}(11), 1119--1129. \url{https://doi.org/10.1016/S0895-4356(00)00242-0}

\leavevmode\vadjust pre{\hypertarget{ref-veveaGeneralLinearModel1995}{}}%
Vevea, J. L., \& Hedges, L. V. (1995). A general linear model for estimating effect size in the presence of publication bias. \emph{Psychometrika}, \emph{60}(3), 419--435. \url{https://doi.org/10.1007/BF02294384}

\leavevmode\vadjust pre{\hypertarget{ref-veveaPublicationBiasResearch2005}{}}%
Vevea, J. L., \& Woods, C. M. (2005). Publication bias in research synthesis: Sensitivity analysis using a priori weight functions. \emph{Psychological Methods}, \emph{10}(4), 428--443. \url{https://doi.org/10.1037/1082-989X.10.4.428}

\leavevmode\vadjust pre{\hypertarget{ref-viechtbauer2010}{}}%
Viechtbauer, W. (2010). Conducting Meta-Analyses in {R} with the metafor Package. \emph{Journal of Statistical Software}, \emph{36}(3). \url{https://doi.org/10.18637/jss.v036.i03}

\leavevmode\vadjust pre{\hypertarget{ref-WickhamDplyr2021}{}}%
Wickham, H., François, R., Henry, L., \& Müller, K. (2021). \emph{Dplyr: A grammar of data manipulation}. \url{https://CRAN.R-project.org/package=dplyr}

\end{CSLReferences}

\end{document}
